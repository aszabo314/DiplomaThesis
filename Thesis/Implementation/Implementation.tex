\chapter{Implementation}
\label{chap:implementation}

This chapter gives an overview of implementation details for this thesis. the implementation of the application is guided by the \textit{functional-first} programming paradigm. \textit{Functional-first} takes inspiration from functional programming, as well as other paradigms, such as object-oriented programming. However, the majority of the application is written purely functional, with connections to non-functional components, such as .NET libraries for multi threading or Schnabel et al.'s shape detection implementation\cite{schnabel-2007-software}. 
\\
This chapter starts with an introduction to functional programming and the programming language \verb|F#|\cite{FSharp} before introducing the key features of the Aardvark platform\cite{aardvark}  in Section \ref{sec:aardvark}. The Aardvark Platform is used as rendering engine for this application. The primary data structure used for organizing the point cloud data is an octree with out-of-core capability. A set of functions that simplify querying  the octree is proposed in Section \ref{sec:funcOctree}. 
\\
Multiple coroutines are performed in the background that produces a stream of data for the point cloud. The multi-threaded architecture of the application is discussed in Section \ref{sec:multithreading}. Section \ref{sec:rendering} concludes this chapter by briefly discussing implementation details of the point-cloud rendering system. 


\section{Functional Programming}
\label{sec:funprog}

Functional programming is a paradigm that views every expression as a mathematical function. The result of each expression is either an elemental data type or a functional type. A core difference to other programming paradigms is the type-level representation of functions (same as data). Functions can be used as input for other functions and have a distinct type defined by its parameters and result. A simple example of a function in \verb|F#|: 

\begin{lstlisting}[language=FSharp]
let square (i : float) : float = i * i 
\end{lstlisting}

This function is of type \verb|float -> float|. It takes a \verb|float| as input and returns a \verb|float|. This function type can be used as input parameter as well. A definition for a function that takes a function as parameter looks as follows: 

\begin{lstlisting}[language=FSharp]
let compute (i: float)(f : float -> float) : float = f i
\end{lstlisting}
Its type is \verb|float -> (float -> float) -> float| and is used as such: 
\begin{lstlisting}[language=FSharp]
let result = compute 10.0 square
\end{lstlisting}

\verb|result| is an expression that executes the \verb|compute| function with arguments $10.0$ of type \verb|float| and \verb|square| of type \verb|float -> float|. Even though this is just a simple example, it showcases the strength of functional programming.  Using functions as input allows the user to create complex and dynamic computations with ease. 
\\

Each expression in a functional context is mathematically defined, parameter space and result space are fully known and must be fully defined on type level. No values that the program does not expect ever occur, thus eliminating undefined behavior. 

Functional programming tries to limit mutation, e.g. change of an external variable's value (program state), as much as possible. A program without mutation is called "`pure"'. Each call to a pure expression with the same parameters must produce the same results. External mutation can change the behavior of expressions that depend on the mutated field, thus changing the result without changing the input. Therefore, avoiding mutation means avoiding undefined behavior.

Finally, strict purity and full type level specification make it possible to reason and construct proofs about programs. This gives the programmer tools to manage very high complexity and reduces the need for debugging significantly.

\\
\verb|F#|\cite{FSharp} is a functional programming language developed and maintained by Microsoft\cite{Microsoft}. F\# is fully integrated into the .NET framework\cite{DotNet} and is built upon the Common Language Infrastructure\cite{CLI}, thus allowing it to use resources written in other languages, such as C\#\cite{CSharp}. Even though F\# is a functional language, it also supports object-oriented programming, such that classes with members can be used as well. 


\section{Aardvark}
\label{sec:aardvark}

The Aardvark platform\cite{aardvark} is a functional-first incremental rendering engine in active development at the VRVis Zentrum für Virtual Reality und Visualisierung\cite{vrvis}. It is an implementation of research results on the field of incremental rendering \cite{worister2013lazy, haaser2015incremental} and semantic shader composition\cite{haaser2014cosmo, haaser2014semantic}. 

The key feature of the Aardvark platform is incremental rendering. In a conventional engine, updates are performed periodically. Each scene object is reevaluated, even though the simulation or user-input may not yield any changes. Incremental rendering counters this overhead by reacting to changes, such that only components that depend on changed values are reevaluated. This section describes some of Aardvark's key features that are used throughout this thesis. Section \ref{sec:adaptive} shows the language-specific constructs on how to build adaptive blocks that react to changes. A \textit{scene graph} represents an object hierarchy in the scene. Section \ref{sec:isg} describes the composition of a scene graph using only a handful of lines of code. 

 
\subsection{Adaptive - IMod - transact}
\label{sec:adaptive}

As mentioned in Section \ref{sec:funprog} mutations are often the cause of undefined behavior and bugs. To introduce state-changeable variables into a functional environment, Aardvark provides the \verb|IMod<'a>| generic type. This generic type is a wrapper around a particular value, whose value might change over time. The programmer does not care about the concrete value, and only specifies the computation applied to the value instead. This lifting of computation makes the interface pure.

Aardvark contains an extensive implementation for a three-dimensional transformation, called \verb|Trafo3d|. The type \verb|IMod<Trafo3d>| listens to changes of the transformation. The following example shows the composition of a model transformation from a position, scale, and rotation, all of which can might their value over time. 

\begin{lstlisting}[language = FSharp]

let position     : IMod<V3d> = ... 
let scale         : IMod<V3d> = ... 
let rotation    : IMod<V3d> = ...

let trafo : IMod<Trafo3d> = 
    adaptive {
    
        let! sc  = scale
        let! rot = rotation
        let! pos = position
        
        
        let S = Trafo3d.scale sc
        let R = Trafo3d.Rotation rot
        let T = Trafo3d.Translation pos
        
        return S * R * T
    }
\end{lstlisting}

The \verb|adaptive| computation expression builder allows the system to keep track of the state of all IMods that are accessed using the \verb|let!| operator. This is called the "`binding operator"', which makes a computation react to their changes. The result of the adaptive block, again, is an IMod<Trafo3d>. If one of the accessed IMods changes its value, the code below, including the \verb|let!| statement, is reevaluated. The computation is kept minimal using caching.
\\

To actively change a value, a subtype of \verb|IMod<'a>|, \verb|ModRef<'a>| is needed. The type \verb|ModRef<'a>| contains the functionality to allow changes to the value, thus causing re-evaluation. All state transactions are collected and executed sequentially, thus reducing the number of re-evaluations and circumventing race-conditions. Each change must be wrapped in a \verb|transact| function, representing transactional logic. 
\\
The following example uses the Aardvark-specific mouse callback function to trigger a reevaluation based on mouse movement. The Move-callback is called each time the mouse moves and provides the user with the old position and the new position. The difference on the x-axis controls the y-value of the rotation. 

\begin{lstlisting}[language = FSharp]
let mouse : IMouse = ...

let rotation : ModRef<V3d> = Mod.init V3d.OOO

mouse.Move.Values.Add(fun (oldPos : PixelPosition, newPos : PixelPosition) -> 
    let delta = 
            newPos.NormalizedPosition.X - oldPos.NormalizedPosition.X
    let angle = delta * 2.0 * Math.Pi
    
    let newRotation = 
            rotation.GetValue() + V3d(0.0, angle, 0.0)
    
    transact(fun () -> Mod.change rotation newRotation)
    )
\end{lstlisting}

The \verb|ModRef| \verb|rotation| can then be bound within an \verb|adaptive| block. 
\\
Transactions are usually used to handle user input. However, asynchronous computations that produce results in parallel use transactions as well to mitigate race conditions on shared data and notify the rest of the program on completion. 


\subsection{Scene graph composition}
\label{sec:isg}

A scene graph (\verb|ISg|) contains all information that is needed to render an object. A common paradigm in functional programming is to split functionality from data, such that the functions that utilize this data are stored in a different namespace than the data. In combination with the pipe operator ($|>$) clean and easy-to-read code can be produced. The following example showcases the composition of an \verb|ISg|. 

\begin{lstlisting}[language = FSharp]

// Transformations
let trafo:    IMod<Trafo3d> = ... 
let view:    IMod<Trafo3d> = ...
let proj:    IMod<Trafo3d> = ...

// IndexedGeometry contains triangles to render
let geometry : IndexedGeometry = ...

// The renderpass for this scenegraph
let renderPass : RenderPass = ...
// The shader (surface) to render the geometry with
let shader : IMod<ISurface> = ...
// Create scenegraph
let sg = geometry   |> Sg.ofIndexedGeometry
                    |> Sg.trafo trafo
                    |> Sg.viewTrafo view
                    |> Sg.projTrafo proj
                    |> Sg.pass renderPass
                    |> Sg.surface shader
\end{lstlisting}

The object \verb|sg| represents a scene graph with transformations, geometry, shader and render pass without the need of complicated constructors or setter functions. Furthermore, an own implementation of \verb|ISg| can easily extend the functionality without reimplemented procedures for this particular type. This extensibility is implemented using a dynamic object attribute mechanism.


\subsection{Out-of-core capabilites}

The Aardvark platform contains functionality to handle out-of-core data. The type \verb|Database| provides means to store data. A chunk of data that is stored on in this database is called \verb|thunk<'a>| where \verb|'a| is the type of the data to be stored. The following code shows how to create a \verb|thunk<'a>|. 

\begin{lstlisting}[language = FSharp]
    let db : Database    = ...
    let data : 'a     = ...
    
    let stored : thunk<'a> = db.ref data
\end{lstlisting}
When a \verb|thunk| is loaded from the database, its content is not loaded into memory yet. Only when accessing the data directly, it is loaded into memory. This is called "`lazy evaluation"'. The data is held in memory as long as the thunk lives. Therefore it is crucial to keep track of unused thunks and dispose of them regularly, thus freeing memory. 


\section{Functional Out-of-core Octree}
\label{sec:funcOctree}

Chapter \ref{chap:octree} already provides information on the capabilities of the octree. This section describes implementation details of the octree's structure using out-of-core mechanics and custom parametric functions for convenient octree queries. 


\subsection{Structure}

The octree's out-of-core capability is managed by an interleaved structure of \verb|thunks|. Each \verb|OctreeNode| contains an array of points. However, the array is not stored directly; it is stored as a \verb|thunk<Point[]>|. Its children are stored in an array of thunks (\verb|thunk<OctreeNode>[]|). The points are stored as a single \verb|thunk|, the reason for storing the node's children in separate thunks is not to load all children into memory when only a subset of children is accessed. Other information such as the centroid, density, rkd tree, and the detected primitive shapes are stored as \verb|thunks| as well. 


\subsection{Elementary functions}

The interaction methods described in Section \ref{sec:interactions} heavily rely on the efficient collection of octree nodes that fulfill certain criteria. Thus, a set of routines is implemented that can easily be parameterized and reused. \verb|F#| collections contain several functions that use a lambda function as input and perform this function on elements in the collection. Some examples are \verb|map, filter| and \verb|choose|, all of which takes two arguments, an collection and a function that takes an element as input. All operations on collections can be composed from these basic functions. This is called the "`mapReduce pattern"'. Even though the octree is not generic, such functions are implemented similarly, to provide a uniform interface to query the octree efficiently. 


\subsubsection{Filter}
The method of filtering returns a subset of elements of the original collection for all of which a decision function returns \verb|true.|
Filtering an octree works similarly. The decision must be made whether or not to use the current node and whether to traverse the children as well, as children may be desired, where the parent is not. 
The function's definition looks as follows: 

\begin{lstlisting}[language = FSharp]
let filter (decisionFun : OctreeNode -> GridCell -> int[] -> bool*bool) (tree: Octree) : (Octree)= ...
\end{lstlisting}

The decision function takes as input the octree's node, grid cell, and unique path. A node's path in the octree is constructed from an array of indices that point to the next predecessor in the tree. It returns a tuple of \verb|bool|, deciding whether to use this node and whether to traverse the children. The result of this operation is a new tree that only contains the nodes that are filtered. For the case that a parent node is not filtered, but some of its children are, an empty placeholder node is used instead to preserve the octree's structure. 


\subsubsection{Map}

Usually, the \verb|map| function is used for generic data structures to create a projection for each element and returns a data structure of the same type with the projected elements. Since the octree is not generic, the \verb|map| function returns the projection for each node of the octree as an array of \verb|'T| where \verb|'T| is the type of the projection. 
The functions's definition looks as follows: 

\begin{lstlisting}[language = FSharp]
let map (projection : OctreeNode -> GridCell -> int[] -> 'T) (tree: Octree) : ('T[])= ...
\end{lstlisting}

Contrary to the filter function, the \verb|map| function does not return a new octree since the type of the projection must not necessarily be hierarchical. Therefore an array of \verb|'T| is returned instead. 


\subsubsection{Choose}

A similar task to filtering is the choose function. However, instead of returning the filtered values directly, the decision function also maps the values to the desired type \verb|'T|. The decision function uses the same input as the decision function for filtering. Its return type is a tuple that consists of a \verb|Option<'T>| and a \verb|bool|. The option type is used to introduce null-mechanics into the functional context. An option can either be \verb|Some value| or \verb|None|. If the \verb|choose| function's returned type is \verb|Some value|, its value is filtered, the \verb|bool| value decides whether or not to traverse the children. 
The function's definition looks as follows: 

\begin{lstlisting}[language = FSharp]
let choose (decisionFun : OctreeNode -> GridCell -> int[] -> Option<'T>*bool) (tree: Octree) :('T[]) = ...
\end{lstlisting}


\begin{figure}[h]
    \centering
    \subcaptionbox{ \label{fig:octreeFilter}}{%
        \includegraphics[width=0.55\textwidth]{Implementation/octreeFilter.png}%7
        }\par\medskip
    \subcaptionbox{ \label{fig:octreeMap}}{%
        \includegraphics[width=0.55\textwidth]{Implementation/octreeMap.png}%
        }\par\medskip        
    \subcaptionbox{ \label{fig:octreeChoose}}{%
        \includegraphics[width=0.55\textwidth]{Implementation/octreeChoose.png}%
        }
    \caption[Filter, map, and choose applied to an exemplary octree]
		{This figure shows an exemplary octree, on which the filter, map and choose function is applied. In (a) the filter function creates a new octree that only contains the nodes that are filtered. (b) shows the result of the map function. Each node in the octree is projected to a new type $t_{i}$ and the result is returned as an array. (c) shows the choose function. Only nodes are returned that fulfill the decision function and are projected to a new type $t_{i}$. }
    \label{fig:octreeFuns}
\end{figure}

Figure \ref{fig:octreeFuns} shows the described functions and its' effects on an exemplary octree. The \verb|filter| function returns a new octree, the\verb|map|  function returns an array of projected types, the \verb|choose| function is a combination of \verb|filter| and \verb|map| such that only those projections are returned that are of interest. 


\subsubsection{Replacing nodes}

As mutations introduce side effects that can lead to bugs, instead of changing information within an octree node, a new node is created containing the new data. The old node might still be in use in a different thread. Thus, race conditions are possible when mutating values. When information changes in an octree node, a new \verb|thunk| is created that stores the new content onto the disc, the old \verb|thunk| is discarded when it is not used anymore. The newly created node has a distinct position that is determined by the path in the octree. The octree is traversed recursively to find the location of the node. 
\\
When resolving the recursion, the content of all ancestor nodes changes as well, since one of the children is a new node. Therefore, for each ancestor, a new node is created too, containing the new content. Since the root changes as well, as the replaced node is a successor of it, a new octree is constructed each time a node is replaced. This octree, however, contains mostly the data of the old octree, except the replaced node. 

\begin{figure}[h]
    \centering
    \includegraphics[width=0.8\textwidth]{Implementation/octreeReplace.png}
    \caption[Example on replacing an octree node]
		{Replacing an octree node subsequently changes the node's ancestors as well. The left tree shows the original octree, the right tree is the new octree after the node [1,0] was replaced. The replaced node is highlighted with a red border, all nodes that changed due to the replacement are colored in orange. }
    \label{fig:octreeReplace}
\end{figure}

Figure \ref{fig:octreeReplace} shows an example on the replacement of a node. The nodes are labeled with their paths in the octree to identify them uniquely. The node with label \verb|[1,2]| is replaced. Thus, all ancestors in the octree are changed as well since the node got a new child. Nodes that changed are colored in orange. 


\subsection{Raycast}

An octree is an ideal data structure to accelerate ray casting. The hierarchical structure ensures that only a minimum set of nodes is tested for intersection. If the ray does not intersect the parent's bounding box, it does not intersect the children's bounding boxes as well. The raycast on the octree can be performed with logarithmic cost. The raycast is implemented using the octree's \verb|choose| function with a decision function that performs the intersection with the bounding box and returns a \verb|RaycastHit| structure containing all necessary information on the raycast hit. 


\subsection{Culling}

Culling utilizes the octree's \verb|filter| function to create a new octree that only contains nodes that are currently rendered. The culling function performs view-frustum culling as well as a \verb|level-of-detail| culling heuristic based on the node's size and distance to the near plane. Figure \ref{fig:octreeCulling} shows an exemplary culling performed on a tree. The \textit{level-of-detail} decreases for nodes that are further away from the camera and only those nodes are used that intersect the view frustum.

\begin{figure}
    \centering
    \includegraphics[width=0.8\textwidth]{Implementation/octreeCulling.png}
    \caption[Exemplary octree culling]
		{An exemplary culling is performed on a tree. Octree nodes that are darker with a thicker edge have a higher \textit{leve-of-detail}. Only close to the camera (red), the nodes with the highest \textit{level-of-detail} are rendered. Nodes that are outside the view frustum(orange) are discarded. }
    \label{fig:octreeCulling}
\end{figure}


Let $V$ be the volume of the node's bounding box, $win_x, win_y$ the dimensions of the window and $d_min$ be the smallest distance from the bounding box to the near plane in world space, clamped by the near and far plane. Furthermore, let $t$ be a user-controlled distance threshold. For this application, $t = 2$ is used. 

\begin{center}
$t_{win} = \frac{t}{max (win_x, win_y)}$ \\
$g = \frac{\sqrt[3]{V}}{d_{min}}$ \\
\end{center}
The decision whether or not to render the node is determined by:
\begin{center}
$g > t_{win}$
\end{center}

The user-controlled threshold $t$ is divided by the largest window dimension to allows the \textit{level-of-detail} the adapt to the window size. $d_min$ is calculated by projecting all corners of the node's bounding box to view space and negating the $z$ value. The smallest value is then clamped to the view frustum, so the computation stays in visible view space. A smaller $t$ win allows for higher \textit{level-of-detail} to be displayed. The granularity $g$ of an octree node is controlled by the minimal distance and the size of the node. Nodes close to the camera have higher the granularity. Smaller nodes have smaller granularity. Figure \ref{fig:targetpointDistance} 

\begin{figure}
    \centering
    \subcaptionbox{ \label{fig:targetpointDistance10}}{%
        \includegraphics[width=0.6\textwidth]{Implementation/targetpointDistance10.png}%7
        }\par\medskip
    \subcaptionbox{ \label{fig:targetpointDistance25}}{%
        \includegraphics[width=0.6\textwidth]{Implementation/targetpointDistance25.png}%
        }\par\medskip        
    \subcaptionbox{ \label{fig:targetpointDistance50}}{%
        \includegraphics[width=0.6\textwidth]{Implementation/targetpointDistance50.png}%
        }
    \caption[Different $t$ values for Technologiezentrum]
		{This figure shows the Technologiezentrum rendered with a $t$ value of 2, 5, and 10 in (a) - (c). A smaller $t$ results in a denser visualization of the point cloud for the current view. }
    \label{fig:targetpointDistance}
\end{figure}


\subsection{Diff}

As discussed in Section \ref{sec:renderHorizon}, the render horizon is the set of nodes that are still rendered, and at least one of the children is not. The \textit{Shape-Assisted Local Level-of-Detail Increment} interaction, described in Section \ref{sec:lod_increment}, uses the render horizon to collect a set of nodes that are beyond the horizon to draw additional points onto the screen. This set of nodes is obtained by the octree's \verb|diff| function. 
\\
Let $t1, t2$ be two octrees where $t2$ is a sub tree of $t1$. The function \verb|diff t1 t2| returns all nodes from t1, whose parents exist in t1 and t2, but the node itself only exists in t1.
\\
The \verb|diff| function takes nodes that are in the first octree and not in the second octree. In the case of the \textit{Shape-Assisted Local Level-of-Detail Increment}, interaction $t1$ is the complete octree, $t2$ is the culled octree that only contains nodes that are rendered. 
\\
Additionally, not only the children are collected, but depending on a depth value, the children's successors are collected as well. 


\section{Sequential Computation Applicator}

The octree receives changes on a regular basis from multiple sources in this application. The shape detection coroutine continuously inserts detected primitive shapes into the octree, resulting in a new octree every time. User interactions change the point set by selecting regions of interest. To synchronize the octree access, locking mechanics can be used. However, such mechanics can get confusing easily, especially when the application grows. 
\\
All changes to the octree are dependent on the state of the octree that, in the meantime, may be altered by an operation from a different thread. Furthermore, since a new octree is created every time, the result of an operation may be overwritten by a second operation that starts while the first operation has not yet finished. All subsequent operations depend on the result of the previous operation. Thus a structure is needed that handles such dependent operations. 
\\
The \verb|Sequential Computation Applicator| is a structure that provides a synchronized way of processing sequential operations. The applicator's functionality is synchronized such that multiple threads can dispatch operations on the octree. The \verb|Sequential Computation Applicator| is a wrapper around a \verb|ModRef<'T>| that invokes changes on the value regularly. 
In this case, the type of the applicator wraps around a \verb|ModRef<Octree>|. 

All dispatched operations are processed sequentially, and the \verb|ModRef|'s value is changed after an operation is completed, thus notifying the Aardvark engine. Since Aardvark evaluates adaptives lazily, operation simply get aggregated as long as no results are requested (i.e. rendering is never stalled). 
The applicator provides the interfaces to dispatch an operation, as well as to dispatch an operation with high priority to enqueue an operation on the front. 

\begin{lstlisting}[language = FSharp]

    member public this.DispatchPrioritized(computation : 'T -> 'T) (timeout : int) (timeoutCallback  : unit -> unit) : unit = ...

    member public this.Dispatch(computation : 'T -> 'T) (timeout : int) (timeoutCallback  : unit -> unit) : unit = ...
   
\end{lstlisting}

The type \verb|'T| is the generic type of the \verb|Sequential Computation Applicator|, in this case, it is \verb|Octree|. A computation is dispatched that takes a \verb|'T|, and projects it to a new \verb|'T| (i.e. all of the previously mentioned computations). The input parameter on execution is the current value of the octree. 
Additionally, an operation can be shut down after a defined timeout in milliseconds. If this is the case the \verb|timeoutCallback| is invoked to allow clean up and error handling. 


\section{Multi-Threaded Environment}
\label {sec:multithreading}

Multithreading can be achieved on multiple scales, depending on the tasks need. The application uses three primary threads that run in parallel. 
\\
The Aardvark rendering engine, combined with the \verb|IMod| evaluation system and user interactions build the main thread of the application. One pitfall of the \verb|IMod| system is that it only reacts to changes in the system, however, to invoke procedures after a particular time without changes is not possible directly (since that would by definition involve mutation). 
\\


The \textit{user-guided shape detection} in Section \ref{sec:user_guided_sd} is triggered when the camera has not changed for a particular amount of time. As the Aardvark \verb|IMod| system only reacts to changes, such no-changes must be invoked by a separate thread that contonously checks for changes. If a shape detection should be performed, this thread dispatches the computation to the applicator thread.
\\
The \textit{sequential computation applicator} performs changes on the octree. All tasks that are dispatched from multiple threads are executed in a sequential order in the background. The thread transacts the changes to the main thread once a computation has finished. 

\begin{figure}[b]
    \centering
    \includegraphics[width=0.6\textwidth]{Implementation/multiThreading.png}
    \caption[Overview on the multi-threaded environment of the application]
		{The main thread controls the rendering, mod evaluation and interactions. Point selections from interactions are handed to the applicator thread. The shape detection is invoked by a separate thread and dispatched on the applicator thread as well.  }
    \label{fig:multiThreading}
\end{figure}

As Figure \ref{fig:multiThreading} shows, multiple threads are dispatching computations to the applicator thread. However, only this thread transacts changes to the main thread. The shape detection invoker thread works independently of the main thread. 
\\

The second parallelization technique used in this thesis is task-based multithreading. The goal is to identify tasks that are executable in parallel for multiple instances of the same type, such as per-point or per-node computations. As long as shared resources are accessed as read-only, tasks can be executed in parallel without race conditions and the need for locking. The input is an array of elements and a mapping function. The mapping function is executed for each element of the input array using the \verb|System.Threading.Tasks.Parallel.For| statement (which is part of .NET). The return type is a new array containing the results of the computation for each element at the original position. The function's signature looks as follows: 
\\
\begin{lstlisting}[language = FSharp]
module Parallel = 
  let map (computation : 'T1->'T2)(array : 'T1[]) : 'T2[]= ...
\end{lstlisting}
The technique is realized as an parallel implementation of F\#s  \verb|map| function for arrays. 

\verb|Parallel.map| is used when point or node conversions are needed, such as projecting points to screen space or calculating the distances of points to a shape. 


\section{Point-Cloud Rendering}
\label{sec:rendering}

Modern point clouds often contain several billion points. Current GPUs are not able to handle such amounts of points. The culling heuristic in Section \ref{sec:renderHorizon} already reduces the octree to set of nodes that can be processed by the GPU and can be drawn in real time. However, the point data may still be located on the hard drive and must be loaded into memory. 
\\
The Aardvark platform already contains a \textit{level-of-detail} point cloud rendering system that can handle out-of-core datasets. The rendering system uses a cache to store nodes that were rendered previously. Once a frame is redrawn, all rendered nodes are collected, and for new nodes, the data is being loaded into the memory. New nodes are added to the cache, nodes that are not used for this frame, are kept in the cache until the memory consumption requires removal of unused data. For each node new to the cache, a queue entry is added for a worker thread to load the point data into memory. The frame is drawn even though not all nodes are loaded. The frame is redrawn once the outstanding data is loaded onto the GPU, thus popping effects occur when the view changes drastically. 
\\
\\
Mathematically, a point has no extent. Therefore it cannot be depicted directly. A common way to draw points is to interpret them as splats of certain size and shape. In this thesis, the points are represented by a sphere imposter. For each point, a camera-aligned quad is created in a geometry shader, whose size is the diameter of the sphere, pixels that are outside the radius of the sphere are discarded. 
\\
Popping effects occur when the view changes and overlapping imposters change their order. Giving the splat a depth displacement that follows a sphere, counters such popping effects. The splat is extruded in world space, such that the depth value follows the curvature of the sphere, causing the imposters to intersect correctly. Figure \ref{fig:point_sprites} shows a direct comparison of using spheres and circular splats. 


\begin{figure}
\centering
\subcaptionbox{ \label{fig:point_circles}}{%
  \includegraphics[width=0.48\textwidth]{Implementation/pointCircles.png}%7
  }
\subcaptionbox{ \label{fig:point_spheres}}{%
  \includegraphics[width=0.48\textwidth]{Implementation/pointSpheres.png}%
  }
  
\caption[Comparison of (a) circular splats and (b) sphere imposters]
{This figure shows a direct comparison of using (a) circular splats and (b) sphere imposters with depth displacement. In (b) the intersections between points are visible due to the spherical shape, whereas points that are little below other points are occluded almost entirely.}
\label{fig:point_sprites}
\end{figure}