% Copyright (C) 2014-2016 by Thomas Auzinger <thomas@auzinger.name>

\documentclass[draft,final]{vutinfth} % Remove option 'final' to obtain debug information.

% Load packages to allow in- and output of non-ASCII characters.
\usepackage{lmodern}        % Use an extension of the original Computer Modern font to minimize the use of bitmapped letters.
\usepackage[T1]{fontenc}    % Determines font encoding of the output. Font packages have to be included before this line.
\usepackage[utf8]{inputenc} % Determines encoding of the input. All input files have to use UTF8 encoding.

% Extended LaTeX functionality is enables by including packages with \usepackage{...}.
\usepackage{amsmath}    % Extended typesetting of mathematical expression.
\usepackage{amssymb}    % Provides a multitude of mathematical symbols.
\usepackage{mathtools}  % Further extensions of mathematical typesetting.
\usepackage{microtype}  % Small-scale typographic enhancements.
\usepackage[inline]{enumitem} % User control over the layout of lists (itemize, enumerate, description).
\usepackage{multirow}   % Allows table elements to span several rows.
\usepackage{booktabs}   % Improves the typesettings of tables.
\usepackage{subcaption} % Allows the use of subfigures and enables their referencing.
\usepackage[ruled,linesnumbered,algochapter]{algorithm2e} % Enables the writing of pseudo code.
\usepackage[usenames,dvipsnames,table]{xcolor} % Allows the definition and use of colors. This package has to be included before tikz.
\usepackage{nag}       % Issues warnings when best practices in writing LaTeX documents are violated.
\usepackage{todonotes} % Provides tooltip-like todo notes.
\usepackage{hyperref}  % Enables cross linking in the electronic document version. This package has to be included second to last.
\usepackage[acronym,toc]{glossaries} % Enables the generation of glossaries and lists fo acronyms. This package has to be included last.
%\usepackage{algorithm2e}
% Define convenience functions to use the author name and the thesis title in the PDF document properties.

\usepackage{listings}
\usepackage{upquote}
 

\usepackage{color}
\definecolor{bluekeywords}{rgb}{0.13,0.13,1}
\definecolor{greencomments}{rgb}{0,0.5,0}
\definecolor{redstrings}{rgb}{0.9,0,0}

\lstdefinelanguage{FSharp}%
{morekeywords={let, new, match, with, rec, open, module, namespace, type, of, member, % 
and, for, while, true, false, in, do, begin, end, fun, function, return, yield, try, %
mutable, if, then, else, cloud, async, static, use, abstract, interface, inherit, finally },
  otherkeywords={ let!, return!, do!, yield!, use!, var, from, select, where, order, by },
  keywordstyle=\color{bluekeywords},
  sensitive=true,
  basicstyle=\ttfamily,
	breaklines=true,
  xleftmargin=\parindent,
  aboveskip=\bigskipamount,
	tabsize=4,
  morecomment=[l][\color{greencomments}]{///},
  morecomment=[l][\color{greencomments}]{//},
  morecomment=[s][\color{greencomments}]{{(*}{*)}},
  morestring=[b]",
  showstringspaces=false,
  literate={`}{\`}1,
  stringstyle=\color{redstrings},
}







\newcommand{\authorname}{Bernhard Rainer} % The author name without titles.
\newcommand{\thesistitle}{Interactive Shape Detection in Out-of-Core Point-Clouds for assisted User Interactions} % The title of the thesis. The English version should be used, if it exists.

% Set PDF document properties
\hypersetup{
    pdfpagelayout   = TwoPageRight,           % How the document is shown in PDF viewers (optional).
    linkbordercolor = {Melon},                % The color of the borders of boxes around crosslinks (optional).
    pdfauthor       = {\authorname},          % The author's name in the document properties (optional).
    pdftitle        = {\thesistitle},         % The document's title in the document properties (optional).
    pdfsubject      = {Subject},              % The document's subject in the document properties (optional).
    pdfkeywords     = {a, list, of, keywords} % The document's keywords in the document properties (optional).
}

\setpnumwidth{2.5em}        % Avoid overfull hboxes in the table of contents (see memoir manual).
\setsecnumdepth{subsection} % Enumerate subsections.

\nonzeroparskip             % Create space between paragraphs (optional).
\setlength{\parindent}{0pt} % Remove paragraph identation (optional).

\makeindex      % Use an optional index.
\makeglossaries % Use an optional glossary.
%\glstocfalse   % Remove the glossaries from the table of contents.

% Set persons with 4 arguments:
%  {title before name}{name}{title after name}{gender}
%  where both titles are optional (i.e. can be given as empty brackets {}).
\setauthor{}{\authorname}{BSc.}{male}
\setadvisor{Pretitle}{Michael Wimmer}{Posttitle}{male}

% For bachelor and master theses:
%\setfirstassistant{Pretitle}{Forename Surname}{Posttitle}{male}
%\setsecondassistant{Pretitle}{Forename Surname}{Posttitle}{male}
%\setthirdassistant{Pretitle}{Forename Surname}{Posttitle}{male}

% For dissertations:
\setfirstreviewer{Pretitle}{Forename Surname}{Posttitle}{male}
\setsecondreviewer{Pretitle}{Forename Surname}{Posttitle}{male}

% For dissertations at the PhD School and optionally for dissertations:
\setsecondadvisor{Pretitle}{Forename Surname}{Posttitle}{male} % Comment to remove.

% Required data.
\setaddress{Heigerleinstraße 53 8, 1170 Wien}
\setregnumber{0828592}
\setdate{1}{03}{2017} % Set date with 3 arguments: {day}{month}{year}.
\settitle{\thesistitle}{\thesistitle} % Sets English and German version of the title (both can be English or German).
%\setsubtitle{Optional Subtitle of the Thesis}{Optionaler Untertitel der Arbeit} % Sets English and German version of the subtitle (both can be English or German).

% Select the thesis type: bachelor / master / doctor / phd-school.
% Bachelor:
%\setthesis{bachelor}
%
% Master:
\setthesis{master}
\setmasterdegree{dipl.} % dipl. / rer.nat. / rer.soc.oec. / master
%
% Doctor:
%\setthesis{doctor}
%\setdoctordegree{rer.soc.oec.}% rer.nat. / techn. / rer.soc.oec.
%
% Doctor at the PhD School
%\setthesis{phd-school} % Deactivate non-English title pages (see below)

% For bachelor and master:
\setcurriculum{Visual Computing}{Visual Computing} % Sets the English and German name of the curriculum.

% For dissertations at the PhD School:
\setfirstreviewerdata{Affiliation, Country}
\setsecondreviewerdata{Affiliation, Country}


\begin{document}

\frontmatter % Switches to roman numbering.
% The structure of the thesis has to conform to
%  http://www.informatik.tuwien.ac.at/dekanat

\addtitlepage{naustrian} % German title page (not for dissertations at the PhD School).
\addtitlepage{english} % English title page.
\addstatementpage

%\begin{danksagung*}
%\todo{Ihr Text hier.}
%\end{danksagung*}

\begin{acknowledgements*}

This thesis would not have been possible without the help of a handful engaged people. 
My first and foremost gratitude goes out to the VRVis Zentrum für Virtual Reality und Visualisierung for providing me with the opportunity to serve an internship and in consequence conduct this thesis. Within the VRVis, I would like to thank the Semantic Modeling and Acquisition (SMAQ) group for integrating me into their team, both on a professional level and amicably. A special thank you goes out to my Aardvark, and F\# Gurus Harald Steinlechner, Georg Haaser, and Attila Szabo, whose help and expertise eased the execution of this thesis enormously. 
I would also like to thank Stefan Maierhofer, leader of the SMAQ group, and my project manager Michael Schwärzler for their ideas and continuous support. 
\\
A big thank you goes out to Michael Wimmer from the Technische Universität Wien for his supervision on not only this thesis but all preceding courses and practica. 

Thanks to Lisa Kellner, who accompanied me through this thesis on a daily basis as she sat next to me and continuously provided me with feedback on my work. 

Thanks to Phillip Erler, a fellow diploma student, for all the exchange of knowledge on both our thesis’s. 

Last but not least, I would like to thank my friends and family that endured my ongoing talks about my thesis for the bigger part of a year and encouraged me to pursue an academic career.  

\end{acknowledgements*}

%\begin{kurzfassung}
%\todo{Ihr Text hier.}
%\end{kurzfassung}

\begin{abstract}

This thesis presents an interactive method of shape detection for out-of-core point clouds. By utilizing the results of the shape detection, improvements are proposed to existing two-dimensional interaction metaphors. Usually, shape detection is performed on the whole point cloud. Thus results are obtained after an extended period of time. Instead, our approach splits the point cloud into chunks of neighboring points that can be processed efficiently in near real-time. The user can select such a region of interest for it to be segmented and is provided with feedback within a fraction of a second, thus making the detected geometry usable almost immediately. The resulting geometries, called primitive shapes, are used to improve existing two-dimensional interaction metaphors. Lasso Selection, as well as Volumetric Brush Selection, benefit from the use of primitive shapes to support the interaction, such that only points are selected that lie roughly follow the curvature of the shape. Thus, the selection of unwanted points in the foreground or background is reduced. Point Picking is improved by the use of a primitive shape as support, as only points are pickable that also belong to the detected primitive shape.\textit{Local Level-of-Detail Increment} is a novel interaction that increases the level-of-detail along a primitive shape, such that additional points that would otherwise be culled are rendered as well. 
\\

As modern point clouds can contain billions of points and the memory capacity of consumer PCs is usually insufficient to store all points at all time, a level-of-detail data structure is used to store the point cloud on the hard disc and data is loaded into memory only on use. This thesis uses a functional octree as data structure to store all points and create a level-of-detail representation of the point cloud. The level-of-detail representation of the point cloud, based on the camera's relative position, is used for interactions and rendering of the point cloud, such that distant regions are presented to the user at lower point resolutions. 
This work was enabled by the Competence Centre VRVis. 

\end{abstract}

% Select the language of the thesis, e.g., english or naustrian.
\selectlanguage{english}

% Add a table of contents (toc).
\tableofcontents % Starred version, i.e., \tableofcontents*, removes the self-entry.

% Switch to arabic numbering and start the enumeration of chapters in the table of content.
\mainmatter


\chapter{Introduction}

\section{Motivation}

With the increasing availability of 3D sensors over the past years, structure finding in point clouds has received increasing interest as well. While different methods for generating point-cloud data exist, such as laser scanners, Microsoft Kinect or photogrammetric reconstructions, the presented data commonly lacks structure and semantic information.Documenting geomorphological erosion, monitoring urban and agricultural developments, mapping archeological sites and generating assets for entertainment industry are only some fields of application for point clouds. 
\\
\\
Point-cloud datasets have grown in size at such a rapid rate that they are now simply too large to fit into system memory, let alone graphics card memory. Therefore, new solutions for out-of-core representations have emerged. In such a case, the point cloud is stored in a cached file on the hard drive and can therefore not be accessed directly. Based on a culling heuristic, chunks of point-cloud data are loaded into memory. This introduces the benefit of only having chunks of data in memory that are of immediate interest to the user. However, the disk speed is a potential bottleneck when it comes to performance because of the continuous swapping of data.
\\
\\
The objective of shape detection is to find similar regions in point clouds with similar characteristics to help the user understand local and global structures. Moreover, it can be used to introduce semantic information into unstructured data, providing the user with more interaction possibilities. Current solutions, such as presented by Schnabel et al. \cite{schnabel-2007-efficient, schnabel-2007-ransac}, can already produce a precise segmentation of point clouds. The complexity of these algorithms increases with the size of the point cloud [// complexity always increases, but by how much (quadratic .. ) ], making the computation for billions of points infeasible in real-time. However, when looking at raw numbers, the approach delivers promising results for point clouds of smaller size (<12.000 points) in a fraction of a second.
While the research fields of representation and rendering of point-clouds, shape detection and segmentation, have received tremendous interest lately, less research focused on interactions and real-time processing. 
\\
This thesis proposes an alternative approach for multi-scale shape detection on local regions in point clouds and presents several interaction techniques that utilize semantic information to assist the user's workflow. This thesis is developed using the Aardvark platform, which is a functional-first incremental rendering engine written in F\#. This thesis showcases some functional aspects of computer graphics as well. 


\section{Problem Definition}

The size of point clouds, obtained by laser scanners or photogrammetric networks, often exceeds several million points. 
Efficient out-of-core solutions for point clouds are discussed in numerous publications, Scheibelbauer \cite{scheiblauer-thesis}, Elseberg et al. \cite{elseberg2013one} or the Point Cloud Library \cite{rusu20113d}, to only name a few. However, a custom solution that stores point clouds, enriched with semantic information, is needed.
\\
In scans of urban environments, many structures can be represented in a more memory-efficient way. Points that follow a wall can often be compressed to few triangles. Pillars often share similarities with cylinders. Detecting such shapes is an immense task that scales with the number of points and fails to be executed in real-time. When exploring a point cloud, immediate feedback of local geometry is useful, since it introduces additional information to the user. This task cannot be achieved without substantial preprocessing of the point cloud. 
\\
The task of precisely selecting regions or points of interest in point clouds can be tedious and cumbersome. By using 2D-interaction metaphors, it is challenging to the user to select spatially neighboring points that follow the same curvature (e.g., points on a wall), as the techniques do not know the desired depth boundaries of the selection region. Interactions across multiple views are needed to achieve such a selection.  Methods that use 3D information, such as a volumetric brush by Weyrich et al. \cite{weyrich2004post}, can make such tasks easier. By consulting the depth buffer each frame, the brush follows the curvature of the geometry. However, by reading pixels from the GPU, the rendering process is stalled. This technique reacts to occlusion such that the brush follows geometry in front of the wall. Thus, view changes are still required.


\section{Contributions}

The main contribution of this diploma thesis is the implementation of an alternative way of detecting shapes on multiple level-of-details in point clouds. Instead of performing shape detection on the complete point cloud at once, we use an approach that lets the user control the region in which shapes should be detected. By reducing those regions to a suitable size, a well-known shape detection algorithm can return meaningful results in interactive time, such that the user is presented with immediate feedback on his selection. 
\\
\\
Shapes are detected on multiple level-of-details in the point cloud. A clustering algorithm takes shapes from different \textit{level-of-detail}, based on a similarity heuristic, and creates a larger connected cluster of shapes. This cluster is used to present information on a larger scale to the user, rather than each shape separately. 
\\
\\ 
This thesis proposes several improvements to commonly known user interactions. \textit{Point Picking} and \textit{Region Selection} are improved by consulting the local geometry of the point cloud to assist the user. By using a shape as support, the interaction dimensions are reduced to the parameter space of the shapes, allowing to exclude unwanted points from interactions easily. 
\\
\\
Additionally,  a novel interaction technique is introduced that allows the user to increment the \textit{level-of-detail} locally along a shape. This helps the user to explore the structure of the point cloud in more detail. 


\section{Structure of the Work}

The structure of the work is as follows. Chapter \ref{chap:related_work} covers the related work for this thesis, including point-cloud rendering and out-of-core representations, shape detection, and segmentation and advanced user interactions on point clouds. Chapter \ref{chap:octree} describes the octree used for the out-of-core representation of the point cloud, as well as some metrics that further describe the content of an octree node. Chapter \ref{chap:shapeDetection} describes the algorithms used to detect primitive shapes in a point cloud and proposes a technique to cluster similar shapes into one coherent shape cluster used for user interactions. Chapter \ref{chap:systemDesign} discusses the application's features including the user-controlled shape detection and assisted interactions that utilize the detected shapes as support shape. Chapter \ref{chap:implementation} focuses on implementation details in a functional context. Results of the application are presented in Chapter \ref{chap:results}. Chapter \ref{chap:conclusion} concludes this thesis with a reflection on the application and an outlook on future work. 
 
\chapter {Related Work}
\section {Out-of-Core Point-Clouds}
\todo{}
\cite{gobbetti2004layered}
\cite{wimmer2006instant}
\cite{wand2007interactive}
\cite{SCHUETZ-2016-POT}
%Potree\cite{SCHUETZ-2016-POT}
%Potree source
\section {Shape Detection}
\label{sec:related_work_shape detection}

The objective of detecting structures in point clouds is a wide field of research. The term structure is very loosely defined. Structure can mean geometric structures, such as planes, cylinder, or spheres. Structure can also be interpreted as more complex components that represent distinct man-made or natural formations, such as cars or lanterns. 
\\
The 2D Hough transform \cite{hough1962method} is a technique usually used in the field of image processing. This method can detect straight lines such as building contours as well as curves. When extending the Hough transform to 3D by Maas et al. \cite{maas1999two} and later by Oda et al. \cite{oda2004automatic} and Overby et al. \cite{overby2004automatic}, the technique can detect planes and other geometric forms such as cylinders \cite{rabbani2005efficient}. 

\begin{figure}
    \centering
    \includegraphics[width=0.81\textwidth]{Related_Work/hough_planes.png}%7
    \caption{Scene created by using 3d Hough transform to detect planes from an airborne laser scanner. Image by Overby et al. \cite{overby2004automatic}.}
    \label{fig:hough_planes}
\end{figure}

Figure \ref{fig:hough_planes} shows a city model consisting of planes, detected by using 3d Hough transform, in a point cloud obtained from airborne laser scanning. Figure \ref{fig:hough_cylinder} showcases a point cloud obtained by a 3d scan and the detected cylinder. 

\begin{figure}
\centering
\subcaptionbox{ \label{fig:picking_raycast }}{
  \includegraphics[width=0.4\textwidth]{Related_Work/hough_cylinder1.png}
  }
\subcaptionbox{ \label{fig:picking_conecast }}{
  \includegraphics[width=0.4\textwidth]{Related_Work/hough_cylinder2.png}
  }
\caption{This figure shows the results of 3d Hough transform used to detect cylinder. (a) shows the input point cloud, (b) shows the detected cylinder. Image by Rabbani et al. \cite{rabbani2005efficient}.}
\label{fig:hough_cylinder}
\end{figure}


A more recent approach by Schnabel et al. \cite{schnabel-2007-efficient} utilizes RANSAC\cite{fischler1981random} to extract a minimal set of primitive shapes that approximate the global structure of the point cloud. The algorithm randomly selects a set of points that roughly follow the curvature of a shape. If a certain number of point follow this shape's curvature as well, the shape is considered to be valid. This approach is capable of detecting planes, cylinders, spheres, cones, and tori and has evolved into one of the most prominent shape detection algorithms. 

\begin{figure}
    \centering
    \includegraphics[width=0.8\textwidth]{Related_Work/schnabel_example.png}%7
    \caption{Left: the original point cloud. Center: The points that belong to a detected shape in random colors. Right: The colors are determined by type (plane = red, cylinder = green, sphere = yellow, cone = purple, torus = grey). Image by Schnabel et al. \cite{schnabel-2007-efficient}. }
    \label{fig:schnabel_church}
\end{figure}

Figure \ref{fig:schnabel_church} shows the results of the RANSAC approach on a point cloud. Not only does this approach provide the geometry of the detected shapes, but it also determines the membership of a point to a shape. This method has evolved into one of the most prominent shape detection algorithms and is used in this thesis as well. 
\\
\\
Tarsha-Kurdi \cite{tarsha2007hough} analyzes the performance of the 3D Hough transform and RANSAC for detecting roof planes from airborne laser data. RANSAC proves to be more robust to noise and more efficient.
\\
\\
GlobFit is a system by Li et al. \cite{li2011globfit} that aims to recover a set of locally fitted primitives, obtained by RANSAC, along with their global mutual relations. The authors work under the assumption that primitives occur repeatedly and are man-made, so global relations are iteratively learned and enforced on the local relations. 
\\
\\
Golvinskiy et al. \cite{golovinskiy2009shape} utilize graph-based methods to recognize shapes in urban environments in 3D point clouds. This method can detect objects, such as cars, newspaper boxes and traffic lights. Potential object locations are identified by clustering nearby points before the point cloud is segmented into foreground and background. For each cluster, a feature vector is built, before using it as input for a trained classifier to obtain a final classification. 
\\
\\
Oesau et al. \cite{oesau2016planar} propose an alternative approach to detect planar shapes in point clouds using region growing. A shape is represented as a set of points and an associated fitting plane. Points or shapes from the neighborhood are added consecutively to the plane, thus growing the region. 
\chapter{Interactions}

Creating new interactions is a key topic for this thesis. This chapter describes the pros and cons of current state-of-the-art two-dimensional interactions and proposes improvements using the detected primitive shapes as interaction support shapes. 

Many proven interaction techniques have emerged over time, such as \textit{Point Picking} or \textit{Region Selection}. 

%% TODO: define pick ray
%% TODO: define candidate global
%% TODO: define point belongs to a shape

\section{Shape Picking}


\section{Point Picking}
\label{sec:picking}
\textit{Point Picking} describes an interaction, where the user is interested in selecting a single point from the scene at a time. A \textit{pick ray} describes a ray originating from the mouse position whose direction is the view direction. The pick radius $r$ denotes the maximum distance of a point to the pick ray in order for the point to be considered a candidate point. Depending on the use case the pick radius $r$ can be depended on the depth value. There are multiple ways of implementing this interaction with varying results. 
\\
\\
The first explored technique is to use a fixed pick radius in world space. The picked point is the point closest to the pick ray in world space. Since the user only interacts with points that are projected onto the nearplane, the projection of the pick radius is smaller for points that lie in the background. Therefore, the distance in pixel between the mouse position and a picked point in the background is smaller than the distance to a picked point in the foreground. While this encourages the picking of points in the foreground, the non-uniform pixel distance introduces inconsistencies. 
\\
\\
A more consistent way of picking a point is to only use the screen space information for each point. The mouse position $p$ in screen space combined with the pick radius $r$ create the pick circle $c$. This circle corresponds to a projection of a cone. All points that intersect this cone are treated as candidate points. In order to calculate this intersection, all points are projected to the screen space. The cone intersects a point if $c$ contains the point in screen space. Then the point with the projection closest to the mouse position is picked. This technique works consistently for different depth values. However, since all points are treated equally, the technique does not distinguish between foreground and background points, thus introducing possible depth ambiguities. 
\\
The projection of points can be executed on the GPU by rendering the projected points, paired with an identifier, to a texture. From this texture, a window around the mouse cursor is downloaded and the closest point is determined. Reading pixels from a texture forces the CPU and GPU to sync and stalls the graphics pipeline. 
\\
\\
The user interacts with points that are presented on the screen only. Moreover, only points are of interest, whose projection on the nearplane lie in close proximity to the mouse position. Since this interaction cannot be computed for all points in real-time, unneeded octree nodes must be filtered beforehand. This prefiltering can easily be achieved by performing a raycast through the octree and collecting all nodes whose bounding boxes intersect the pick ray. However, consider the case, that the pick ray does not intersect a node's bounding box, but the distance of the box to the ray is smaller than the pick radius. Some points might exist that should be considered candidates, but due to the nature of a raycast, are discarded. This introduces the possibility that points that can be the picking result, are not considered, introducing inconsistency to the pick interaction. One solution to overcome this problem is to use a conecast instead. 
\\
A circle on the nearplane is the projection of a cone in world space. The corners of the box are projected onto the nearplane and the convex hull polygon is calculated. The intersection then is determined by the intersection of the polygon with the pick circle $c$. 


\subsection{Shape-Assisted Point Picking}
Picking comes with the disadvantage that some constellations of points can influence the picking interaction in a negative way. Points that occlude structures of interest force the user to change the view in order to pick the desired point. In some cases, a point in the background is favored over a desired point on a structure. 
 \textit{Shape-Assisted Point Picking} utilizes primitive shapes to perform the picking routine only on points that are part of a structure. The user selects a cluster of shapes, thus reducing the amount of possible candidate points to only those that belong to this shape. 
\\
Instead of performing a cone- or raycast on the octree, only those nodes are taken into account, whose bounding boxes intersect the shape cluster. Each point that does not fulfill the score function from Section \ref{sec:scorefun} for the particular canidate shape is discarded as well, leaving only a handful of points on which a conecast is performed. The pick radius in world space is calculated by unprojecting the pick circle to the intersection point of the pick ray with the shape cluster. Only points are considered that lie in the pick sphere, constructed by the intersection point and the pick radius. The point closest to the intersection point is then picked. Due to the curvature of shapes, such as cylinders and spheres, points on the back of a shape are projected in close proximity to the mouse position as well. By using the projected distances, points that lie on the back side of the shape might get favored over points that are on the front side of the shape (facing the user). 
\\

This technique comes not only with interaction benefits, computation time is drastically reduced as well. Usually a shape cluster consists of less nodes than a raycast since the cluster's extension is limited to a region in the point cloud. Points within a node are also reduced such that intersections and distance measures are computed only for candidate points. 

\\
Figure \ref{fig:picking} shows the different picking methods, described in Section \ref{sec:picking}. Figure \ref{fig:picking_raycast} showcases a simple raycast with a radius. The combination of a ray and a radius yields a cylinder, which contains all candidate points on world space. The pick distance in world space is consistent. Figure \ref{fig:picking_conecast} uses a conecast instead. The opening angle is defined by the pick radius in screen space. The pick distance in world space increases the higher the depth value. All points inside the volume are treated equally, introducing consistency in screen space. \Figure\ref {fig:picking_assisted} showcases the use of a support shape to further filter candidate points. All points are filtered that belong to the support shape prior to be used as input for a spherecast. 

\begin{figure}
\centering
\subcaptionbox{ \label{fig:picking_raycast}}{%
  \includegraphics[width=0.6\textwidth]{Interactions/picking_raycast.png}%7
  }\par\medskip
\subcaptionbox{ \label{fig:picking_conecast}}{%
  \includegraphics[width=0.6\textwidth]{Interactions/picking_conecast.png}%
  }\par\medskip        
\subcaptionbox{ \label{fig:picking_assisted}}{%
  \includegraphics[width=0.6\textwidth]{Interactions/picking_assisted.png}%
  }
\caption{Two-dimensional illustration of various picking methods. Candidate points are colored in green, other points are colored in red. The areas in red describe the different volumes in which candidate points are located. (a) showcases a picking process using a simple raycast. The ray combined with a radius constructs a cylinder in world space which contains all candidate points, (b) uses a cone instead. (c) utilizes a selected shape (dark blue) in order to further filter the candidate points to only follow the curvature of the shape. A spherecast is then performed on the filtered points using the unprojected pick radius as radius to select the final set of candidate points. }
\label{fig:picking}

\end{figure}


\section{Region Selection}
Region Selection aims to not pick a single point at once, but select a set of points, that are spatial neighbors
The design for the \textit{Shape-Assisted Region Selection} is guided by one seemingly simple example task: \textit{Select all points that belong to this wall}. A wall can intersect with other building elements such as roof, balconies or the ground. In regions close to intersections, it is tedious and cumbersome to only select points on the desired structure. Using two-dimensional interaction metaphors, selecting spatially neighboring points along the same curvature, is particularly challenging, since the system does not know the desired depth boundaries for the selection region. In this chapter the benefits of using support shapes for two- and three-dimensional interaction metaphor are discussed. 

\subsection{Volumetric Brush}

The \textit{Volumetric Brush} by Weyrich et. al\cite{weyrich2004post} is designed in such a way that a volume is projected onto the foremost geometry. Points that intersect this volume are considered to be selected. To retrieve the projected position of the volume, usually a sphere, the depth buffer is consulted and the depth value for the current mouse position is retrieved. The world position is the unprojection of the mouse position's $xy$-coordinates and the depth value. 
\\
Since this technique follows the foremost geometry only, sudden depth changes occur if the area of interest is occluded by different geometry. Thus view changes are still required to achieve the example task. In regions close to intersections with other structures, such as below the roof, the user must control the size of the volume in order to not select points on neighboring structures. 

\subsection{Lasso}

The \textit{Lasso} is a common two-dimensional interaction metaphor. The user draws a polygon onto the screen. All points, whose projection lie inside the polygon are selected. Much like \textit{Point Picking}, points are projected onto the nearplane and and the intersection between the point and the polygon determines if the point is selected. Since this interaction is plain two-dimensional, the selection creates a volume that extrudes the polygon along the view direction. All points that lie within this volume are selected. 

% TODO
%%The task of precisely selecting regions or points of interest in point clouds can be tedious and cumbersome. Using 2D-interaction metaphors only, it is challenging to
%%select spatially neighboring points that follow the same curvature (e.g, points on a wall), as the system does not know the desired depth boundaries of the selection %%region. Interactions across multiple views are needed to achieve such a selection. Methods that use three-dimensional information, such as a volumetric brush by Weyrich %%et. al\cite{weyrich2004post}, can ease such tasks. By consulting the depth buffer each frame the brush follows the curvature of the geometry. 



\subsection{Shape-Assisted Region Selection}
\section{Shape-Assisted Local Level-of-Detail Increase}

\chapter{Out-of-core Octree}
\label{chap:octree}


\section{Overview}

Modern point clouds are often too large to fit into memory, let alone video memory. The term \textit{out-of-core} is used to describe the management of datasets whose size exceed the available system memory. 
A large-scale point cloud is stored in a structured file, such that chunks of data can be loaded into memory efficiently. A structure that partitions data into usable chunks that share spatial properties is an octree. 
An octree is a hierarchical datastructure in which each node represents a spatial region, defined by a three-dimensional bounding box. If the decision is made to split a node, eight children are created, each representing an octant of the parent's bounding box.


\section{Node Split Ruling}

Numerous decision rules exist that determine whether a node is split or not. A node is partitioned if the point count exceeds a threshold $n$.  A decision based on the number of points in a node is favored. 
Having a consistent number of points in an octree node allows for little variation in loading time and data size. With nodes of consistent size, the runtime of procedures for different nodes is similar as well. Therefore, such methods can be used in a context where immediate feedback is useful, such as user interactions, since the execution time can be estimated. In this thesis, a point count $n = 5000$ is chosen. 
\\
\\
If a node is partitioned, its elements are replaced by a random subset of points of size $n$ from its children, thus creating an efficient \textit{level-of-detail} representation of the point cloud on multiple scales. 
Other implementations, such as Potree \cite{SCHUETZ-2016-POT}, where some points remain in the parent instead of being stored in a child node, are more efficient regarding disc space. However, in this application, each node can be viewed as self-contained, such that no points from predecessor nodes are needed to represent the point cloud for this region and \textit{level-of-detail} entirely. 


\section{Out-of-core Functionalities}

\textit{Out-of-core} describes a way of handling datasets that are too large to fit into system memory. Small chunks of data are loaded into memory on demand, while a significant part of the dataset remains on the hard drive. If an octree node is loaded into memory, its content, points and child nodes, remains on the disc until needed. Thus, quick interactions can be performed with an octree node without the need of loading the content into memory. 
Loading data into memory can only be done as long as free memory is available. Unused chunks of data are removed from memory after not access happened for a certain amount of time. 


\section{Octree Postprocessing}

Point clouds often only contain information on position and color and lack distinct geometric features such as normal vectors. Normal vectors are significant for shape detection as they introduce information on the local curvature to the point cloud.  After the octree build process is completed, additional properties are computed to enrich the dataset with additional information. 


\subsection{rkd-Tree}

An rkd-tree by Tobler \cite{tobler2011rkd} is an efficient datastructure for point queries. While the octree partitions the space into somewhat small regions, the rkd-tree is used to find distinct points within a node at a specified location quickly. 


\subsection{Normals}

For lighting and shape detection, each point must possess a normal vector. The local neighborhood determines a point's normal. Using the node's rkd-tree a $k$-nearest-neighbor search is performed to get the $k$ closest neighbors. Principal Component Analysis \cite{jolliffe2002principal} is used to fit a plane into the neighborhood whose normal vector is used as the point's normal.


\subsection{Centroid}

The centroid of a node provides an indicator of the distribution of points in the octree node. The centroid is used as a target for the camera to focus on the presumably most dense part of the point cloud. 


\subsection{Density}

The density describes the average distance between a point and its nearest neighbor. The density increases with higher \textit{level-of-detail} since more points are contained in a smaller region. To find the nearest neighbor, again, the node's rkd-tree is used. 


\section{Octree Culling and Render Horizon}
\label{sec:renderHorizon}

As a point cloud contains more data than the GPU can render, only points from nodes are drawn that contribute to the currently viewed scene. The result of the culling operation is a new octree that contains only nodes that are currently rendered. 
\\
\\
A simple, yet powerful culling heuristic is view frustum culling. Nodes that are outside of the view frustum are discarded. By using view frustum culling, whole branches of the octree are removed. However, the remaining branches are still too large to be rendered completely. A \textit{level-of-detail} decision function determines if a node should be rendered. Depending on the node's distance to the nearplane and its volume a decision is made if the node should be rendered or not. 

\begin{figure}
    \centering
    \includegraphics[width=0.5\textwidth]{Octree/renderHorizon.png}
    \caption[Illustration of the render horizon of an octree.]
		{The render horizon of an octree is showcased as a dashed line. Nodes that are rendered, are colored in turquoise, nodes that are part of the render horizon, are bordered in magenta. }
    \label{fig:renderHorizon}
\end{figure}


The \textit{Render horizon} describes a cut that separates the octree into a rendered part and an unrendered part. The nodes of which some of the children are affected by the cut create the render horizon. Nodes along the render horizon share the property that one of its direct children contains points that are are not rendered anymore. This property is useful when additional [ // what? ] information should be displayed. 
\\
Figure \ref{fig:renderHorizon} showcases an example of a cut, displayed as a dashed line, through an octree along with its created render horizon. Rendered nodes are colored in turquoise. A dashed line originating from a node indicates that the node contains children who are not rendered. Nodes whose border is colored in magenta belong to the set of nodes of the render horizon. These nodes contain edges that are intersected by the cut. 
\\
As the rendered parts of the octree are needed for multiple tasks in the application, culling is performed only once per frame. The result is an octree that only contains the nodes that are currently rendered. Thus, interactions that rely on visible data must only be performed on this octree without performing culling operations twice. 
\section {Shape Detection}
\label{sec:related_work_shape detection}

The objective of detecting structures in point clouds is a wide field of research. The term structure is very loosely defined. Structure can mean geometric structures, such as planes, cylinder, or spheres. Structure can also be interpreted as more complex components that represent distinct man-made or natural formations, such as cars or lanterns. 
\\
The 2D Hough transform \cite{hough1962method} is a technique usually used in the field of image processing. This method can detect straight lines such as building contours as well as curves. When extending the Hough transform to 3D by Maas et al. \cite{maas1999two} and later by Oda et al. \cite{oda2004automatic} and Overby et al. \cite{overby2004automatic}, the technique can detect planes and other geometric forms such as cylinders \cite{rabbani2005efficient}. 

\begin{figure}
    \centering
    \includegraphics[width=0.81\textwidth]{Related_Work/hough_planes.png}%7
    \caption{Scene created by using 3d Hough transform to detect planes from an airborne laser scanner. Image by Overby et al. \cite{overby2004automatic}.}
    \label{fig:hough_planes}
\end{figure}

Figure \ref{fig:hough_planes} shows a city model consisting of planes, detected by using 3d Hough transform, in a point cloud obtained from airborne laser scanning. Figure \ref{fig:hough_cylinder} showcases a point cloud obtained by a 3d scan and the detected cylinder. 

\begin{figure}
\centering
\subcaptionbox{ \label{fig:picking_raycast }}{
  \includegraphics[width=0.4\textwidth]{Related_Work/hough_cylinder1.png}
  }
\subcaptionbox{ \label{fig:picking_conecast }}{
  \includegraphics[width=0.4\textwidth]{Related_Work/hough_cylinder2.png}
  }
\caption{This figure shows the results of 3d Hough transform used to detect cylinder. (a) shows the input point cloud, (b) shows the detected cylinder. Image by Rabbani et al. \cite{rabbani2005efficient}.}
\label{fig:hough_cylinder}
\end{figure}


A more recent approach by Schnabel et al. \cite{schnabel-2007-efficient} utilizes RANSAC\cite{fischler1981random} to extract a minimal set of primitive shapes that approximate the global structure of the point cloud. The algorithm randomly selects a set of points that roughly follow the curvature of a shape. If a certain number of point follow this shape's curvature as well, the shape is considered to be valid. This approach is capable of detecting planes, cylinders, spheres, cones, and tori and has evolved into one of the most prominent shape detection algorithms. 

\begin{figure}
    \centering
    \includegraphics[width=0.8\textwidth]{Related_Work/schnabel_example.png}%7
    \caption{Left: the original point cloud. Center: The points that belong to a detected shape in random colors. Right: The colors are determined by type (plane = red, cylinder = green, sphere = yellow, cone = purple, torus = grey). Image by Schnabel et al. \cite{schnabel-2007-efficient}. }
    \label{fig:schnabel_church}
\end{figure}

Figure \ref{fig:schnabel_church} shows the results of the RANSAC approach on a point cloud. Not only does this approach provide the geometry of the detected shapes, but it also determines the membership of a point to a shape. This method has evolved into one of the most prominent shape detection algorithms and is used in this thesis as well. 
\\
\\
Tarsha-Kurdi \cite{tarsha2007hough} analyzes the performance of the 3D Hough transform and RANSAC for detecting roof planes from airborne laser data. RANSAC proves to be more robust to noise and more efficient.
\\
\\
GlobFit is a system by Li et al. \cite{li2011globfit} that aims to recover a set of locally fitted primitives, obtained by RANSAC, along with their global mutual relations. The authors work under the assumption that primitives occur repeatedly and are man-made, so global relations are iteratively learned and enforced on the local relations. 
\\
\\
Golvinskiy et al. \cite{golovinskiy2009shape} utilize graph-based methods to recognize shapes in urban environments in 3D point clouds. This method can detect objects, such as cars, newspaper boxes and traffic lights. Potential object locations are identified by clustering nearby points before the point cloud is segmented into foreground and background. For each cluster, a feature vector is built, before using it as input for a trained classifier to obtain a final classification. 
\\
\\
Oesau et al. \cite{oesau2016planar} propose an alternative approach to detect planar shapes in point clouds using region growing. A shape is represented as a set of points and an associated fitting plane. Points or shapes from the neighborhood are added consecutively to the plane, thus growing the region. 
\section{Shape Matching}
\label{sec:shapeMatching}

Since shape detection is performed on multiple scales and shapes are extracted for small regions at a time, a heuristic is needed to determine if shapes from different nodes belong to the same structure. While on a low \textit{level-of-detail}, a wall may be contained by a single octree node, on a higher level, a node will only contain parts of the wall and multiple nodes contain primitive shapes that share this wall. This section presents a set of matching functions to determine if two shapes originate from the same geometry with a certain tolerance. Only shapes of the same type can be matching. Therefore it is not necessary to define functions to match e.g. a plane shape with a cone shape since the result will always be \verb|false|. 


\subsection{Elementary Matching Functions}
\label{sec:elementarMatchingFuns}

As the primitive shapes are represented by only a handful of parameters, it is sufficient to determine a similarity measure (called "`matching"') of these parameters. Therefore, we firstly define elementary matching functions for numbers, vectors, positions, and axes. Since matching is an approximation of equality, each matching function's result is determined by comparing a computation result to a deviation threshold $\epsilon$. If a matching function returns \verb|true| (i.e. the values don't deviate too far), the input values are considered to be matching. 

\begin{itemize}
    \item \textbf{Matching floats $f_1, f_2$}: 
        $$\frac{f_1}{f_2} \geq \epsilon, \textrm{ where } f_1 \leq f_2$$  
    \item \textbf{Matching vectors $v_1, v_2$}: 
        $$\frac{v_1}{|v_2|} \cdot \frac{v_2}{|v_2|} \geq \epsilon$$
    \item \textbf{Matching positions $p_1, p_2$}: 
        $$\sqrt{(p_1.x - p_2.x)^2 + (p_1.y - p2_y)^2 + (p1_z - p2_z)^2} \leq \epsilon$$
    \item \textbf{Matching axis $a_1, a_2$}: 
    \\
    An axis is defined by a start and end point. Let $p_{01},p_{02}$ be the start and end point of $a_1$ and $p_{11}, p_{12}$ the start and end point of $a_2$. Furthermore, let $v_1, v_2$ be the direction vectors of $a_1$ and $a_2$. The rays of the axes are denoted as $r_1 = p_{00} + sv_1$ and $r_2 = p_{10} + tv_2$. The closest distances for start and end point for each axis to the complementary ray are calculated. From those for values, the largest value $d$ is used for decision making. The matching decision is composed as follows: 
        $$\frac{v_1}{|v_2|} \cdot \frac{v_2}{|v_2|} \geq \epsilon_1 \land d \leq \epsilon_2$$
\end{itemize}


\subsection{Primitive Shape Matching Functions}
\label{sec:primitiveShapeMatchingFuns}

With the elementary matching functions defined in Section \ref{sec:elementarMatchingFuns}, it is easy to define matching functions for two primitive shapes based on the elementary matching functions:

\begin{itemize}
\item \textbf{Matching plane shapes}: 
A plane shape contains to a quad that encloses all support points. For distance computation, the plane is used, rather than the quad, since the origin of the plane shape is the plane. For each corner of each quad, the distance to the other plane is calculated. From those eight values, the largest distance $d$ is chosen. Two plane shapes are matching, if the planes' normal vectors are matching in retrospect to a threshold value $\epsilon_1$ and $d$ is smaller than or equal to a threshold value $\epsilon_2$.
\item \textbf{Matching cylinder shapes}: 
A cylinder consists of an axis and a radius. Two cylinder shapes are matching if radii and axes are matching. 
\item \textbf{Matching cone shapes}:
Cones consist of an axis, an apex, and an opening angle. Two cone shapes are matching if the axes, apexes and opening angles are matching. 
\item \textbf{Matching sphere shapes}: 
Two sphere shapes are matching if the center positions and the radii are matching. 
\item \textbf{Matching torus shapes}: 
A torus consists of a center position, an axis and a major and minor radius. Two torus shapes are matching if the center position, axes, major radii, and minor radii are matching. 
\end{itemize}

The matching result heavily depends on the chosen threshold values. Table \ref{tab:matchingThresholds} shows the $\epsilon$ values that are used for this implementation. Plane matching uses a custom heuristic that is not depicted in the table. For this heuristic, $\epsilon_2 = 0.05$ is chosen. Note that matching floats is a relative measure, whereas matching positions and axes uses world space distances as the threshold. Matching vectors uses the angle between the two vectors to calculate a matching. 

\begin{table}
\centering
\begin{tabular}{ r | r }
    Matching    & threshold values \\
    \hline
  Floats         & $\epsilon = 0.99$ \\
    Vectors     & $\epsilon = 0.95$ \\
  Positions & $\epsilon = 0.05$ \\ 
    Axis             & $\epsilon_2 = 0.05$, $\epsilon_1 = 0.95$ \\  

\end{tabular}
\caption[Different threshold values for parameter matching]
{The different threshold values for parameter matching.}
\label{tab:matchingThresholds}
\end{table}


\section{Shape Clustering}
\label{sec:shapeClustering}

With the user interaction in mind, it is not sufficient to interact with a single shape only. To determine the global extent of a primitive shape, more information is needed. Section \ref{sec:primitiveShapeMatchingFuns} describes a set of heuristics to determine if two primitive shapes originate from the same geometry. These heuristics are used in this section in order to create larger representations of a primitive shape. 
\textit{Shape Clustering} aims to find matching primitive shapes for a base shape and build a larger coherent cluster of primitive shapes over multiple level-of-details [// achtung: level of detail richtig zu verwenden. es gehört so: ein level of detail, mehrere levels of detail oder detail levels. bitte bessere das überall aus]. The types of primitive shapes can be categorized as finite and infinite shapes. Spheres and tori are finite objects per definition, thus it is sufficient enough to find a set of matching primitive shapes to create a cluster. For infinite shapes, such as planes, cones, and cylinder, additional computations are needed in order to create a coherent shape cluster. Figure \ref{fig:cuboids} shows the case of two cuboids, whose front face share a plane. By only using the matching functions to create a cluster, both front faces are packed into the cluster, even tough there is a visible gap between them.

\begin{figure}
    \centering
    \includegraphics[width=0.8\textwidth]{Shape_Detection/cuboids.png}
    \caption[Point cloud consisting of two cuboids.]
		{A generic point cloud of two cuboids. The detected planes are rendered in grey.}
    \label{fig:cuboids}
\end{figure}

Section \ref{sec:matchingSetBuilding} describes the procedure of finding matching shapes in the octree. For infinite shapes, an additional step is performed using a region-growing approach described in Section \ref{sec:regionGrowing}. 


\subsection{Building a set of matching primitive shapes}
\label{sec:matchingSetBuilding}

In order to find matching shapes, the octree is searched for primitive shapes that match the base shape. Only octree nodes are searched that are currently rendered, such that the cluster only consists of shapes that are present in the scene. Already, the cluster consists of shapes that share the same geometry as origin. For sphere and torus shapes, this step is sufficient enough to create a valid cluster, since both shapes are finite. 


\subsection{Graph-based Region Growing}
\label{sec:regionGrowing}

A cluster of shapes can be seen as an $\epsilon$-connected component from a larger graph. A complete graph is created using all matching shapes as vertices. In a complete graph, an edge exists for any pair of vertices. The weight of an edge is determined by a custom distance function for each primitive shape:

\begin{itemize}
    \item \textbf{Plane Shape}:         As a plane shape is bounded by a quad, the distance between two plane shapes is computed as the shortest distance between the two bounding quads.
    \item \textbf{Cylinder Shape}:    The shortest distance between two cylinder shapes is determined by the shortest distance of all pairs of start/end points of both cylinders. 
  \item \textbf{Cone Shape}:            The shortest distance between two cone shapes is determined by the shortest distance of all pairs of start/end points from both cones. 
\end{itemize}


A cluster is created by growing a region in a graph, adding only vertices that connect to the current region via an edge, whose weight is smaller than $\epsilon$. This creates a cluster of shapes, ensuring that the distance to the closest neighboring shape is at maximum $\epsilon$. 
\\
\begin{figure}
    \centering
    \includegraphics[width=0.5\textwidth]{Shape_Detection/regionGrowingPlanes.png}
    \caption[Exemplary $\epsilon$-connected plane cluster]
		{A cluster of plane shapes created by computing the $\epsilon$-connected component. Planes that belong to the cluster are colored in turquoise.}
    \label{fig:regionGrowingPlanes}
\end{figure}

Figure \ref{fig:regionGrowingPlanes} shows an exemplary illustration on region growing for plane shapes. The distance between two plane shapes is measured as the closest distance between the two bounding quads. 

\begin{figure}
\centering
\subcaptionbox{ \label{fig:regionGrowingCylinder}}{%
  \includegraphics[width=0.3\textwidth]{Shape_Detection/regionGrowingCylinder.png}%7
  }
\subcaptionbox{ \label{fig:regionGrowingCone}}{%
  \includegraphics[width=0.5\textwidth]{Shape_Detection/regionGrowingCone.png}%
  }
    \caption[Exemplary cylinder and clone cluster]
		{This figure shows a cylinder cluster and a cone cluster build from matching shapes. Shapes that belong to the cluster are colored in turquoise.}
    \label{fig:regionGrowingConeCylinder}
\end{figure}

Figure \ref{fig:regionGrowingConeCylinder} showcases region growing for both, cylinder shapes and cone shapes. The matching heuristic already confirms that the shapes lie on the same axis and share a similar radius. Therefore, instead of a three-dimensional world distance, a one-dimensional distance between two points is sufficient to build a shape cluster. 
\\
\\
The region growing component of the clustering heavily depends on the $\epsilon$ distance threshold. A proper distance threshold that mirrors the region's topology well is the density of an octree node as described in Section \ref{sec:shapeDetectionParameterSelection}. For this task, we chose the density of the node of the base shape, more specific: $\epsilon = density \cdot 2.0$.
\chapter{System Design}
\label{chap:systemDesign}

This chapter discusses the design of the application developed for this thesis. The aim is to develop a stand-alone desktop application to view and interact with out-of-core point clouds. The application lets the user not only view point clouds but also interact with it on multiple levels. The application has three main elements that interact together, all controlled by the user. 
\\
Shape detection is performed one small region at a time. Section \ref{sec:user_guided_sd} describes an interactive heuristic that selects an octree node as a region of interest, using the user's cursor position and camera view as input. Using the user's input effectively changes the task of detecting shapes from an automated approach to an user-controlled interaction.
\\
Section \ref{sec:shapePicking} describes a picking algorithm to select a primitive shape from the octree that can later be used as support shape for assisted user interactions. 
\\
Section \ref{sec:interactions} provides detailed information on the additional user interactions that use a support shape as assistance. \textit{Point Picking, Region Selection, and Local Level-of-Detail Increment} are all interactions that benefit from the utilization of a support shape. 
\\
Most interactions presented in this thesis share similar techniques and naming conventions. Therefore, some terms that are used throughout this chapter are defined in Section \ref{sec:termDefinitions}. 


\section{Term Definitions}
\label{sec:termDefinitions}

The base of all interactions is the user's cursor on the screen. The \textit{pick ray} is a ray that goes from the cursor’s position in world space in the view direction. 
Each interaction iteratively filters the octree's data as such that coarse filtering is carried out before finer adjustments are performed on the dataset before choosing a final candidate. Data that survives the coarse filtering is referred to as \textit{candidate} data, e.g. candidate nodes or candidate points.  [// eine skizze wäre hier sehr gut]


\section{User-guided Shape Detection}
\label{sec:user_guided_sd}

Shape Detection by Schnabel et al. \cite{schnabel-2007-efficient} usually is performed on the whole point cloud at once. However, this computation might take several minutes, depending on the size of the point cloud. The aim of this thesis is to implement a method such that the user is provided with feedback for a region immediately. Therefore, the point cloud is processed in chunks of roughly the same size instead of as a whole. The octree already provides the point cloud as pieces of spatial-neighboring data, such that a node describes an enclosed subset of the point cloud at a specific \textit{level-of-detail}. Furthermore, using chunks of the same size, allows the user to expect results in about the same time for each node. This response time should ideally be a fraction of a second, at best $0-250$ milliseconds. 

The \textit{User-guided shape detection} is performed continuously in the background, relying only on the user's current mouse position. Thus, only nodes are considered that intersect the pick ray. To be selected as candidate node from the octree, a node must fulfill the following constraints: 
\begin{itemize}
    \item The node must intersect the pick ray.
    \item The node must currently be rendered and visible to the user. 
    \item The node must contain at least $n$ points, the same amount used as minimal support point count for shape detection.
    \item The node must not already contain detected shapes.
\end{itemize}

Since the user guides the shape detection, it makes sense that the user only interacts with what is presented on screen. Therefore the node must currently be rendered and visible. To reduce the amount of redundant computation, only nodes that contain enough points to fit at least one shape, are considered to be candidates. Lastly, the shape detection algorithm works under the optimistic assumption that, once it terminates, all shapes in this region are detected. Therefore, nodes that already contain detected shapes do not qualify as candidates as well. 
\\
The culling operation on the octree, described in Section \ref{sec:renderHorizon}, returns an octree that only contains nodes that are rendered. Thus, all nodes in this tree already are visible to the user. Only a single raycast must be performed on the culled octree to obtain the set of candidate nodes. 
\\
All nodes from the raycast result are eliminated that do not fulfill the previous constraints. The heuristic favors nodes with higher \textit{level-of-detail}, such that the user receives geometric information for the most detailed parts of the currently explored region first. The projected distance to the nearplane is used to select between nodes with the same \textit{level-of-detail}. Algorithm ~\ref{alg:candidateNodeHeuristic} showcases the selection heuristic that is executed for each node that intersects the pick ray. The most suitable node for shape detection is the node with the highest \textit{level-of-detail}, closest to the nearplane. 

\begin{algorithm}
    
    \KwData{raycastResults : OctreeNode[]}
    \KwResult{node : OctreeNode}
    node     = null\;
    level = 0\;
    npd     = 1.0\;
\ForEach{currentNode \normalfont{\textbf{in}} raycastResults}
    {            
        currentNpd = calculateDistanceToNearPlane(node)    \;
        currentLevel = node.getLodLevel()\;
        \If{currentLevel > level \normalfont{\textbf{or}} (currentLevel == level \normalfont{\textbf{and}} currentNpd < npd)}{
            node $\leftarrow$ currentNode\;
            level $\leftarrow$ currentLevel\;
            npd $\leftarrow$ currentNpd\;                            
        }
    }
\caption[A heuristic that chooses the most suitable candidate node]{selectCandidateNode}
\label{alg:candidateNodeHeuristic}
\end{algorithm}

The selection of a suitable candidate node depends heavily on the camera's position. When zooming out, the camera moves away from the scene, thus reducing the render horizon and therefore reducing the maximum \textit{level-of-detail}. By ranking different nodes that fulfill the candidate criteria, even when the view does not change, a multi-scale representation of the local geometry is constructed over time, creating a \textit{level-of-detail} for primitive shapes as well. 


\section{Shape Picking}
\label{sec:shapePicking}

This thesis presents several interactions that are supported by detected shapes to ease interactions with point clouds.  
However, this raises the need for an additional interaction to pick a primitive shape. A raycast is performed on the octree using the pick ray. Only those nodes are filtered that contain primitive shapes. Furthermore, only those shapes are used that intersect the pick ray. The picking heuristic prefers primitive shapes of higher \textit{level-of-detail} that are closer to the camera. All primitive shapes are collected from the candidate nodes and sorted using a custom key. Primitive shapes that intersect the pick ray come with information on the nearest intersection point and the \textit{level-of-detail}. For the intersection points, the $depth$ is calculated. The sort key is composed as follows: 

$$key = level + 1 - depth$$

The primitive shape with the highest key is chosen as support shape. The composition of the key ensures that a shape with the highest \textit{level-of-detail} that is closest to the camera is picked.
\\
As a primitive shape only spans over a small local region and user interactions usually are performed on larger areas, a single primitive shape is not sufficient to provide semantic information to the user. Therefore, once a shape is picked, a cluster is created from this shape using primitive shapes that are similar to this shape. This clustering is described in Section \ref{sec:shapeClustering} in depth. Shape clustering creates a homogenous cluster of shapes from different octree nodes. The cluster has the property that each of its shapes has a neighbor within $\epsilon$,
 as described in Chapter\ref{chap:shapeDetection}. Thus, a cluster can be seen as a single, larger shape that is referred to as \textit{support shape}.


\section{Shape-Assisted Interactions}
\label{sec:interactions}

Creating new interactions is a key topic in this thesis. Classic two-dimensional interaction metaphors, such as \textit{Lasso Selection} and \textit{Point Picking} are limited due to the lack of information on the desired depth. Therefore those interactions must either guess the desired depth boundaries or ignore them. Furthermore, such interactions lack the possibility to mask out unwanted points from a selection. Unwanted points are selected on a regular basis. Hence, many view changes are necessary to select points that are of interest only. Using a primitive shape as support, the user can easily interact with the point cloud, such that only points are considered which belong the support shape. 
\\
Sections \ref{sec:pointPicking} and \ref{sec:regionSelection} describe the pros and cons of current state-of-the-art two-dimensional interactions and propose improvements using a detected primitive shape as support shape. Furthermore, Section \ref{sec:lod_increment} proposes a technique to locally increment the rendered \textit{level-of-detail} along structures of interest to amplify details. 


\subsection{Shape-Assisted Point Filtering}
\label{sec:pointFiltering}

The set of candidate nodes can be created in numerous ways, depending on the task. However, once a task uses a cluster as support shape, the filtering of points works similarly. The set of candidate nodes is further reduced by calculating the intersection of the node's bounding box with the cluster. For each node, the point set is reduced as well. A point belongs to the shape if it fulfills the score function for this particular shape as described in Section \ref{sec:scorefun}. All points that do not belong to this shape are discarded, thus creating a set of points that only consists of support points of the shape cluster. This set of points can then be used for various interactions, described in the following sections. 


\subsection{Point Picking}
\label{sec:pointPicking}

\textit{Point Picking} describes an interaction, where the user is interested in selecting a single point from the scene at a time. A \textit{pick ray} describes a ray originating from the mouse position whose direction is the view direction. The pick radius $r$ denotes the maximum distance of a point to the pick ray for the point to be considered a candidate point. Depending on the use case the pick radius $r$ can be depended on the depth value. There are multiple ways of implementing this interaction with varying results. 
\\
\\
The first explored technique is to use a fixed pick radius in world space. The picked point is the point closest to the pick ray in world space. Since the user only interacts with points that are projected onto the nearplane, the projection of the pick radius is smaller for points that lie in the background. Therefore, the distance in pixels from the mouse position to the picked point in the background is smaller than the distance to a picked point in the foreground. While this encourages the picking of points in the front, the non-uniform pixel distance introduces inconsistencies as the cursor reacts stronger to points in the foreground. 
\\
\\
A more consistent way of picking a point is to only use the screen space information for each point. The mouse position $p$ in screen space combined with the pick radius $r$ create the pick circle $c$. This circle corresponds to a projection of a cone. All points that intersect this cone are treated as candidate points. To calculate this intersection, all points are projected to the screen space. The cone intersects a point if $c$ contains the point in screen space. Then the point with the projection closest to the mouse position is picked. This technique works consistently for different depth values. However, since all points are treated equally, the method does not distinguish between foreground and background points, thus introducing possible depth ambiguities. 
\\
The projection of points can be executed on the GPU by rendering the projected points, paired with an identifier, to a texture. From this texture, a window around the mouse cursor is downloaded, and the closest point is determined. Reading pixels from a texture forces the CPU and GPU to sync and stalls the graphics pipeline. 
\\ [// auch hier wären skizzen super]
\\
The user interacts with points that are presented on the screen only. Moreover, only points are of interest, whose projection on the nearplane lie in proximity to the mouse position. Since this interaction cannot be computed for all points in real-time, unneeded octree nodes must be filtered beforehand. This pre-filtering can easily be achieved by performing a raycast through the octree and collecting all nodes whose bounding boxes intersect the pick ray. However, consider the case that the pick ray does not intersect a node's bounding box, but the distance of the box to the ray is smaller than the pick radius. Some points might exist that should be considered candidates, but due to the nature of a raycast, are discarded. Therefore, the possibility exists that points that can be the picking result, are not considered, introducing inconsistency to the pick interaction. One solution to overcome this problem is to use a conecast instead. 
\\
A circle on the nearplane is the projection of a cone in world space. The corners of the box are projected onto the nearplane, and the convex hull polygon is calculated. The intersection then is determined by the intersection of the polygon with the pick circle $c$. 


\subsubsection{Shape-Assisted Point Picking}
\label {sec:picking_assisted}

Picking comes with the disadvantage that some constellations of points can influence the picking interaction negatively. The user is forced to change the view to pick an otherwise occluded point from the structure of interest. In some cases, a point in the background is favored over the desired point on a structure in the foreground. \textit{Shape-Assisted Point Picking} utilizes primitive shapes to perform the picking routine only on points that are part of a structure. The user selects a cluster of shapes, thus reducing the number of possible candidate points to only those that belong to this support shape. 
\\ 
Instead of using a cone- or raycast to collect all candidate nodes, only nodes whose bounding boxes intersect the support shape are of interest. Furthermore, only points are considered that belong to the support shape. The intersecting points of the octree nodes and are filtered as described in Section \ref{sec:pointFiltering}, leaving only a handful of nodes and points on which the interaction is performed. 
\\
\subsubsection{Backface elimination}
\\
Due to shapes possibly having front and back sides, such as cylinders and spheres, points on the back of a shape are projected near the mouse position as well. By using the projected distances, points that lie on the back side of the shape might get favored over points that are on the front side of the shape (facing the user). 
\\
Therefore, the point picking is performed in world space using a pick sphere. The sphere's position is the intersection point of the pick ray with the support shape. The sphere's radius is calculated by unprojecting the pick radius to the intersection point. Only points are considered that lie in the pick sphere, constructed by the intersection point and the pick radius. The point closest to the intersection point is then selected. 
\\
\subsubsection{Performance}
\\
This technique comes not only with interaction benefits; computation time is reduced as well. Usually, a shape cluster intersects fewer nodes than a raycast result since the cluster's extension is limited to a region in the point cloud. The number of points per node is reduced as well, and distance measures are computed only for candidate points. 
\\
\begin{figure}
\centering
\subcaptionbox{ \label{fig:picking_raycast}}{%
  \includegraphics[width=0.6\textwidth]{System_Design/picking_raycast.png}%7
  }\par\medskip
\subcaptionbox{ \label{fig:picking_conecast}}{%
  \includegraphics[width=0.6\textwidth]{System_Design/picking_conecast.png}%
  }\par\medskip        
\subcaptionbox{ \label{fig:picking_assisted}}{%
  \includegraphics[width=0.6\textwidth]{System_Design/picking_assisted.png}%
  }
\caption
[Illustration on different picking methods. (a) shows a simple raycast, (b) a cone cast, (c) shows the use of a support shape combined with a sphere cast]
{Two-dimensional illustration of various picking methods. Candidate points are colored in green; other points are colored in red. The areas in red describe the different volumes in which candidate points are located. (a) showcases a picking process using a simple raycast. The ray combined with a radius constructs a cylinder in world space which contains all candidate points, (b) uses a cone instead. (c) utilizes a selected shape (dark blue) to filter candidate points that follow the curvature of the shape. A sphere cast is then performed on the filtered points using the unprojected pick radius to obtain the final set of candidate points. }
\label{fig:picking_overview}
\end{figure}


Figure \ref{fig:picking_overview} shows the different picking methods, described in Section \ref{sec:pointPicking}. Figure \ref{fig:picking_raycast} showcases a simple raycast with a radius. The combination of a ray and a radius yields a cylinder, which contains all candidate points on world space. The pick distance is consistent in world space. Figure \ref{fig:picking_conecast} uses a conecast instead. The opening angle is defined by the pick radius in screen space. The pick distance in world space increases the higher the depth value. All points inside the volume are treated equally, introducing consistency in screen space. Figure \ref{fig:picking_assisted} showcases the use of a support shape to filter candidate points. All points that belong to the support shape are filtered prior to being used as input for a spherecast. 


\begin{figure}
    \centering
    \includegraphics[width=0.8\textwidth]{System_Design/picking_assisted_screenshot.png}%7
    \caption[Screenshot of Shape-assisted Point Picking]
		{\textit{Shape-assisted Point Picking} is performed on a shape cluster that represents the wall in the foreground. The cross hair indicates the picked point. Note that the cross hair does not jump to a point in the background even though they would be closer to the cursor in screen space. }
    \label{fig:picking_assisted_screenshot}
\end{figure}

The result of \textit{Shape-assisted Point Picking} can be seen in Figure \ref{fig:picking_assisted_screenshot}. The crosshair, even though other points' projections are closer to the cursor, sticks to the structure. Picking points on edges is improved in particular since the picking result follows the edge rather than jumping to a point in the background. 


\subsection{Region Selection}
\label{sec:regionSelection}

Region Selection aims to select a set of points, that share certain criteria, rather than picking a single point. 
The design for the \textit{Shape-Assisted Region Selection} is guided by one seemingly simple example task: \textit{Select points that belong to this wall only}. A wall can intersect with other building elements such as roof, balconies or the ground. In regions close to intersections, it is tedious and cumbersome to only select points on the desired structure. Using two-dimensional interaction metaphors, selecting multiple spatially neighboring points along the same curvature is particularly challenging due to the system not knowing the desired depth boundaries for the selection region. In this section, the benefits of using support shapes for two- and three-dimensional interaction metaphors to select regions of points,  are discussed. 


\subsubsection{Lasso Selection}

The \textit{Lasso Selection} is a common two-dimensional interaction metaphor used for multiple geometry-based applications. While it is a useful technique to selected regions in 2D, drawbacks appear when porting the interaction to 3D. The user draws a polygon onto the screen. All points whose projection lie inside this polygon are selected. Much like \textit{Point Picking}, points are projected onto the nearplane, the intersection between the point and the lasso polygon in screen space determines whether a point is selected. The lasso polygon in combination with the camera view creates a three-dimensional volume, whose area contains all points whose projection lie inside the lasso polygon. Figure \ref{fig:lasso_sketch} showcases the volume created by a lasso polygon drawn onto the screen.

Note that this interaction is computed asynchronously and the selection is performed on the complete octree. Therefore, it is essential that octree nodes, whose \textit{level-of-detail} are too high and therefore are not rendered, are still included in this interaction as well. 


\begin{figure}
    \centering
    \includegraphics[width=0.8\textwidth]{System_Design/lasso_sketch.png}%7
    \caption[Illustration of the creation of a lasso selection]
		{The user draws a polygon(red) on the screen(light blue). The constructed three-dimensional area(yellow) contains all points, whose projection lie inside the lasso polygon. }
    \label{fig:lasso_sketch}
\end{figure}


\begin{figure}
\centering
\subcaptionbox{ \label{fig:lasso1}}{%
  \includegraphics[width=0.5\textwidth]{System_Design/lasso1.png}%7
  }\par\medskip
\subcaptionbox{ \label{fig:lasso2}}{%
  \includegraphics[width=0.5\textwidth]{System_Design/lasso2.png}%
  }\par\medskip        
\subcaptionbox{ \label{fig:lasso3}}{%
  \includegraphics[width=0.5\textwidth]{System_Design/lasso3.png}%
  }
\caption[Screenshots of the workflow of a lasso selection. (a) shows the lasso, (b) the selected points, (c) shows the selected points from a different angle. ]
{(a) - (c) show a lasso selection performed on a point cloud. In (a) the user draws a polygon onto the screen. In (b) the selected points are visualized in red. Figure (c) showcases the selection from a different angle. All points that are projected to the area of the polygon are selected.The unintentional selection of points that are obscured by objects in the foreground is a byproduct of the \textit{Lasso Selection}. }
\label{fig:lasso}
\end{figure}


Figure \ref{fig:lasso} shows a lasso selection performed on a point cloud. The user draws a polygon onto the screen. The selected points are highlighted in red. When changing the view, selected points that were occluded while drawing the lasso, appear. The user must control the selection distance by hand to minimize this effect. However, to solve the task of only selecting points on the wall, further lasso selections must be applied to remove points from the selection that were selected unintentionally. 


\subsubsection{Shape-Assisted Lasso Selection}

The aim of this interaction is to provide a smaller set of points on which a \textit{Lasso Selection} is performed.  The octree nodes are again filtered using the technique described in Section \ref{sec:pointFiltering} to reduce the number of candidate nodes and points. On this reduced set, a \textit{Lasso Selection} is performed. The result of this interaction is a selection that mimics a \textit{Lasso Selection}, with the benefit of not selecting 'through' the point cloud. The depth ambiguities of the \textit{Lasso Selection} are circumvented by introducing continuous depth boundaries defined by the local curvature of the shape cluster. 

Figure \ref{fig:lasso_assisted} shows the workflow for selecting points on a shape. The shape is selected beforehand by the user. A lasso is drawn on the screen which selects all points that lie within the lasso and belong to the selected support shape. In Figure \ref{fig:lasso_assisted3}, it can be seen that contrary to Figure\ref{fig:lasso3}, no points are selected that do not belong to the support shape.

\begin{figure}
\centering
\subcaptionbox{ \label{fig:lasso_assisted1}}{%
  \includegraphics[width=0.5\textwidth]{System_Design/lasso_assisted1.png}%7
  }\par\medskip
\subcaptionbox{ \label{fig:lasso_assisted2}}{%
  \includegraphics[width=0.5\textwidth]{System_Design/lasso_assisted2.png}%
  }\par\medskip        
\subcaptionbox{ \label{fig:lasso_assisted3}}{%
  \includegraphics[width=0.5\textwidth]{System_Design/lasso_assisted3.png}%
}
\caption[Screenshots of the workflow of a Shape-Assisted Lasso Selection. (a) shows the lasso and the support shape, (b) the selected points, (c) shows that only points are selected on the support shape.]
{(a) - (c) show a \textit{Shape-Assisted Lasso Selection} performed on a point cloud. The front facing wall is selected as support shape by the user. The shape cluster is visualized in light red. In (a) the user draws a polygon onto the screen after selecting a shape as support shape. (b) shows the selected points visualized in red from the same point of view. Upon view change, it can be seen that this interaction only selects points that belong the shape cluster. }
\label{fig:lasso_assisted}
\end{figure}


\subsubsection{Volumetric Brush}

The \textit{Volumetric Brush} by Weyrich et al. \cite{weyrich2004post} is designed in such a way that a volume is projected onto the foremost geometry. Points that intersect this volume are considered to be selected. To retrieve the projected position of the volume, usually a sphere, the depth buffer is consulted, and the depth value for the current mouse position is retrieved. The world position is the unprojection of the mouse position's $xy$-coordinates and the depth value. 
\\
Since this technique follows the foremost geometry only, sudden depth changes occur if different geometry occludes the area of interest. Thus, view changes are still required to achieve the example task. In regions close to intersections with other structures, such as below the roof, the user must control the size of the volume to not select points on neighboring structures. 


\subsubsection{Shape-Assisted Volumetric Brush}

The \textit{Volumetric Brush} can easily be adapted to be used in combination with support shapes. Instead of consulting the depth buffer to reconstruct the cursor’s world position, the pick ray is intersected with the selected support shape, thus resulting in a three-dimensional world position. The selection is then performed only on the filtered set of points as described in Section \ref{sec:pointFiltering}. 
Figure \ref{fig:brush} shows a \textit{Shape-Assisted Volumetric Brush} interaction performed on a wall. The wall creates as a cluster of planes. Only points are selected that belong to this cluster. 

\begin{figure}
\centering
\subcaptionbox{ \label{fig:brush_assisted1}}{%
  \includegraphics[width=0.8\textwidth]{System_Design/brush1.png}%7
  }\par\medskip
\subcaptionbox{ \label{fig:brush_assisted2}}{%
  \includegraphics[width=0.8\textwidth]{System_Design/brush2.png}%
  }
\caption[Workflow of the Shape-Assisted Volumetric Brush. (a) shows the trajectory of the brush, (b) shows the selected points. ]
{This figure shows a \textit{Shape-Assisted Volumetric Brush} selection performed shape cluster (transparent red) detected in a point cloud. In (a) the trajectory of the brush is shown as subsequently rendered spheres(grey). (b) shows the selected points for this brush interaction. Even tough some points of the roof structure are intersecting the brush, they are not selected. }
\label{fig:brush}
\end{figure}


\subsection{Shape-Assisted Local Level-of-Detail Increment}
\label{sec:lod_increment}

To further investigate the local structures of a point cloud, the currently rendered maximum level-of-detail might not suffice. The highest \textit{level-of-detail} is chosen such that the GPU is not overloaded and the balance between detail and performance is retained. However, temporarily adding a handful of additional nodes is sufficient to provide more detailed information, and doesn't pose an enormous impact on performance.
\\

Section \ref{sec:renderHorizon} describes the set of nodes that are still rendered, but where some of the children are not rendered anymore, as \textit{render horizon}. 
Increasing the \textit{level-of-detail} is only useful for nodes that lie on the \textit{render horizon}. Beyond these nodes, more detailed information exists that is currently not rendered. A raycast is performed on the render horizon. All nodes that intersect the ray are considered to be candidate nodes. The successor nodes of the candidate nodes hold more detailed information on the region of interest. Depending on a \textit{level-of-increment} parameter, controlled by the user, the successors of those nodes are added to the scene and rendered. A \textit{level-of-increment} value of $1$ results in adding all children for each node on the render horizon, a parameter of $2$ results in adding all children's children to the scene. 
By adding smaller nodes of higher level-of-detail, the overall detail in the scene is amplified. However, by plainly adding additional nodes, noise and unwanted structures are amplified as well. 
\\

\textit{Shape-Assisted Local Level-of-Detail Increment} utilizes a selected shape cluster to amplify detail only on structures of interest. The user selects a primitive shape as support shape. Instead of performing a raycast, the nodes on the render horizon that intersect the support shape, are collected. From these nodes, only those points that belong to the support shape, are added to the scene. The membership of a point to the shape cluster is determined by the score function described in Section \ref{sec:scorefun}. 

\begin{figure}
\centering
\subcaptionbox{ \label{fig:lod_increment1}}{%
  \includegraphics[width=0.8\textwidth]{System_Design/lod_Increase.png}%7
  }\par\medskip
\subcaptionbox{ \label{fig:lod_increment2}}{%
  \includegraphics[width=0.8\textwidth]{System_Design/lod_Increase2.png}%
  }
\caption[Comparison of a scene with and without Shape-Assisted Level-of-Detail Increment.]
{This figure shows the benefits of \textit{Shape-Assisted Level-of-Detail Increment} for a point cloud. (a) shows the currently rendered structure, as well as as shape cluster that is currently selected (red). (b) shows the scene with amplified details along the structure. The additional points are rendered without lighting to appear brighter. [// die bilder gehören vergrößert - das rundherum wegschneiden und nur das in der mitte groß zeigen]}
\label{fig:lod_increment}
\end{figure}

Figure \ref{fig:lod_increment} showcases the difference in a scene with- and without amplified details along a wall. The selected support shape is rendered in transparent red, while the additional points are rendered without lighting to appear brighter. In the intersecting octree nodes, only those points are rendered that fulfill the shape's score function. 



\chapter{Implementation}
\label{chap:implementation}

This chapter gives an overview of implementation details for this thesis. the implementation of the application is guided by the \textit{functional-first} programming paradigm. \textit{Functional-first} takes inspiration from functional programming, as well as other paradigms, such as object-oriented programming. However, the majority of the application is written purely functional, with connections to non-functional components, such as .NET libraries for multi threading or Schnabel et al.'s shape detection implementation\cite{schnabel-2007-software}. 
\\
This chapter starts with an introduction to functional programming and the programming language \verb|F#|\cite{FSharp} before introducing the key features of the Aardvark platform\cite{aardvark}  in Section \ref{sec:aardvark}. The Aardvark Platform is used as rendering engine for this application. The primary data structure used for organizing the point cloud data is an octree with out-of-core capability. A set of functions that simplify querying  the octree is proposed in Section \ref{sec:funcOctree}. 
\\
Multiple coroutines are performed in the background that produces a stream of data for the point cloud. The multi-threaded architecture of the application is discussed in Section \ref{sec:multithreading}. Section \ref{sec:rendering} concludes this chapter by briefly discussing implementation details of the point-cloud rendering system. 


\section{Functional Programming}
\label{sec:funprog}

Functional programming is a paradigm that views every expression as a mathematical function. The result of each expression is either an elemental data type or a functional type. A core difference to other programming paradigms is the type-level representation of functions (same as data). Functions can be used as input for other functions and have a distinct type defined by its parameters and result. A simple example of a function in \verb|F#|: 

\begin{lstlisting}[language=FSharp]
let square (i : float) : float = i * i 
\end{lstlisting}

This function is of type \verb|float -> float|. It takes a \verb|float| as input and returns a \verb|float|. This function type can be used as input parameter as well. A definition for a function that takes a function as parameter looks as follows: 

\begin{lstlisting}[language=FSharp]
let compute (i: float)(f : float -> float) : float = f i
\end{lstlisting}
Its type is \verb|float -> (float -> float) -> float| and is used as such: 
\begin{lstlisting}[language=FSharp]
let result = compute 10.0 square
\end{lstlisting}

\verb|result| is an expression that executes the \verb|compute| function with arguments $10.0$ of type \verb|float| and \verb|square| of type \verb|float -> float|. Even though this is just a simple example, it showcases the strength of functional programming.  Using functions as input allows the user to create complex and dynamic computations with ease. 
\\

Each expression in a functional context is mathematically defined, parameter space and result space are fully known and must be fully defined on type level. No values that the program does not expect ever occur, thus eliminating undefined behavior. 

Functional programming tries to limit mutation, e.g. change of an external variable's value (program state), as much as possible. A program without mutation is called "`pure"'. Each call to a pure expression with the same parameters must produce the same results. External mutation can change the behavior of expressions that depend on the mutated field, thus changing the result without changing the input. Therefore, avoiding mutation means avoiding undefined behavior.

Finally, strict purity and full type level specification make it possible to reason and construct proofs about programs. This gives the programmer tools to manage very high complexity and reduces the need for debugging significantly.

\\
\verb|F#|\cite{FSharp} is a functional programming language developed and maintained by Microsoft\cite{Microsoft}. F\# is fully integrated into the .NET framework\cite{DotNet} and is built upon the Common Language Infrastructure\cite{CLI}, thus allowing it to use resources written in other languages, such as C\#\cite{CSharp}. Even though F\# is a functional language, it also supports object-oriented programming, such that classes with members can be used as well. 


\section{Aardvark}
\label{sec:aardvark}

The Aardvark platform\cite{aardvark} is a functional-first incremental rendering engine in active development at the VRVis Zentrum für Virtual Reality und Visualisierung\cite{vrvis}. It is an implementation of research results on the field of incremental rendering \cite{worister2013lazy, haaser2015incremental} and semantic shader composition\cite{haaser2014cosmo, haaser2014semantic}. 

The key feature of the Aardvark platform is incremental rendering. In a conventional engine, updates are performed periodically. Each scene object is reevaluated, even though the simulation or user-input may not yield any changes. Incremental rendering counters this overhead by reacting to changes, such that only components that depend on changed values are reevaluated. This section describes some of Aardvark's key features that are used throughout this thesis. Section \ref{sec:adaptive} shows the language-specific constructs on how to build adaptive blocks that react to changes. A \textit{scene graph} represents an object hierarchy in the scene. Section \ref{sec:isg} describes the composition of a scene graph using only a handful of lines of code. 

 
\subsection{Adaptive - IMod - transact}
\label{sec:adaptive}

As mentioned in Section \ref{sec:funprog} mutations are often the cause of undefined behavior and bugs. To introduce state-changeable variables into a functional environment, Aardvark provides the \verb|IMod<'a>| generic type. This generic type is a wrapper around a particular value, whose value might change over time. The programmer does not care about the concrete value, and only specifies the computation applied to the value instead. This lifting of computation makes the interface pure.

Aardvark contains an extensive implementation for a three-dimensional transformation, called \verb|Trafo3d|. The type \verb|IMod<Trafo3d>| listens to changes of the transformation. The following example shows the composition of a model transformation from a position, scale, and rotation, all of which can might their value over time. 

\begin{lstlisting}[language = FSharp]

let position     : IMod<V3d> = ... 
let scale         : IMod<V3d> = ... 
let rotation    : IMod<V3d> = ...

let trafo : IMod<Trafo3d> = 
    adaptive {
    
        let! sc  = scale
        let! rot = rotation
        let! pos = position
        
        
        let S = Trafo3d.scale sc
        let R = Trafo3d.Rotation rot
        let T = Trafo3d.Translation pos
        
        return S * R * T
    }
\end{lstlisting}

The \verb|adaptive| computation expression builder allows the system to keep track of the state of all IMods that are accessed using the \verb|let!| operator. This is called the "`binding operator"', which makes a computation react to their changes. The result of the adaptive block, again, is an IMod<Trafo3d>. If one of the accessed IMods changes its value, the code below, including the \verb|let!| statement, is reevaluated. The computation is kept minimal using caching.
\\

To actively change a value, a subtype of \verb|IMod<'a>|, \verb|ModRef<'a>| is needed. The type \verb|ModRef<'a>| contains the functionality to allow changes to the value, thus causing re-evaluation. All state transactions are collected and executed sequentially, thus reducing the number of re-evaluations and circumventing race-conditions. Each change must be wrapped in a \verb|transact| function, representing transactional logic. 
\\
The following example uses the Aardvark-specific mouse callback function to trigger a reevaluation based on mouse movement. The Move-callback is called each time the mouse moves and provides the user with the old position and the new position. The difference on the x-axis controls the y-value of the rotation. 

\begin{lstlisting}[language = FSharp]
let mouse : IMouse = ...

let rotation : ModRef<V3d> = Mod.init V3d.OOO

mouse.Move.Values.Add(fun (oldPos : PixelPosition, newPos : PixelPosition) -> 
    let delta = 
            newPos.NormalizedPosition.X - oldPos.NormalizedPosition.X
    let angle = delta * 2.0 * Math.Pi
    
    let newRotation = 
            rotation.GetValue() + V3d(0.0, angle, 0.0)
    
    transact(fun () -> Mod.change rotation newRotation)
    )
\end{lstlisting}

The \verb|ModRef| \verb|rotation| can then be bound within an \verb|adaptive| block. 
\\
Transactions are usually used to handle user input. However, asynchronous computations that produce results in parallel use transactions as well to mitigate race conditions on shared data and notify the rest of the program on completion. 


\subsection{Scene graph composition}
\label{sec:isg}

A scene graph (\verb|ISg|) contains all information that is needed to render an object. A common paradigm in functional programming is to split functionality from data, such that the functions that utilize this data are stored in a different namespace than the data. In combination with the pipe operator ($|>$) clean and easy-to-read code can be produced. The following example showcases the composition of an \verb|ISg|. 

\begin{lstlisting}[language = FSharp]

// Transformations
let trafo:    IMod<Trafo3d> = ... 
let view:    IMod<Trafo3d> = ...
let proj:    IMod<Trafo3d> = ...

// IndexedGeometry contains triangles to render
let geometry : IndexedGeometry = ...

// The renderpass for this scenegraph
let renderPass : RenderPass = ...
// The shader (surface) to render the geometry with
let shader : IMod<ISurface> = ...
// Create scenegraph
let sg = geometry   |> Sg.ofIndexedGeometry
                    |> Sg.trafo trafo
                    |> Sg.viewTrafo view
                    |> Sg.projTrafo proj
                    |> Sg.pass renderPass
                    |> Sg.surface shader
\end{lstlisting}

The object \verb|sg| represents a scene graph with transformations, geometry, shader and render pass without the need of complicated constructors or setter functions. Furthermore, an own implementation of \verb|ISg| can easily extend the functionality without reimplemented procedures for this particular type. This extensibility is implemented using a dynamic object attribute mechanism.


\subsection{Out-of-core capabilites}

The Aardvark platform contains functionality to handle out-of-core data. The type \verb|Database| provides means to store data. A chunk of data that is stored on in this database is called \verb|thunk<'a>| where \verb|'a| is the type of the data to be stored. The following code shows how to create a \verb|thunk<'a>|. 

\begin{lstlisting}[language = FSharp]
    let db : Database    = ...
    let data : 'a     = ...
    
    let stored : thunk<'a> = db.ref data
\end{lstlisting}
When a \verb|thunk| is loaded from the database, its content is not loaded into memory yet. Only when accessing the data directly, it is loaded into memory. This is called "`lazy evaluation"'. The data is held in memory as long as the thunk lives. Therefore it is crucial to keep track of unused thunks and dispose of them regularly, thus freeing memory. 


\section{Functional Out-of-core Octree}
\label{sec:funcOctree}

Chapter \ref{chap:octree} already provides information on the capabilities of the octree. This section describes implementation details of the octree's structure using out-of-core mechanics and custom parametric functions for convenient octree queries. 


\subsection{Structure}

The octree's out-of-core capability is managed by an interleaved structure of \verb|thunks|. Each \verb|OctreeNode| contains an array of points. However, the array is not stored directly; it is stored as a \verb|thunk<Point[]>|. Its children are stored in an array of thunks (\verb|thunk<OctreeNode>[]|). The points are stored as a single \verb|thunk|, the reason for storing the node's children in separate thunks is not to load all children into memory when only a subset of children is accessed. Other information such as the centroid, density, rkd tree, and the detected primitive shapes are stored as \verb|thunks| as well. 


\subsection{Elementary functions}

The interaction methods described in Section \ref{sec:interactions} heavily rely on the efficient collection of octree nodes that fulfill certain criteria. Thus, a set of routines is implemented that can easily be parameterized and reused. \verb|F#| collections contain several functions that use a lambda function as input and perform this function on elements in the collection. Some examples are \verb|map, filter| and \verb|choose|, all of which takes two arguments, an collection and a function that takes an element as input. All operations on collections can be composed from these basic functions. This is called the "`mapReduce pattern"'. Even though the octree is not generic, such functions are implemented similarly, to provide a uniform interface to query the octree efficiently. 


\subsubsection{Filter}
The method of filtering returns a subset of elements of the original collection for all of which a decision function returns \verb|true.|
Filtering an octree works similarly. The decision must be made whether or not to use the current node and whether to traverse the children as well, as children may be desired, where the parent is not. 
The function's definition looks as follows: 

\begin{lstlisting}[language = FSharp]
let filter (decisionFun : OctreeNode -> GridCell -> int[] -> bool*bool) (tree: Octree) : (Octree)= ...
\end{lstlisting}

The decision function takes as input the octree's node, grid cell, and unique path. A node's path in the octree is constructed from an array of indices that point to the next predecessor in the tree. It returns a tuple of \verb|bool|, deciding whether to use this node and whether to traverse the children. The result of this operation is a new tree that only contains the nodes that are filtered. For the case that a parent node is not filtered, but some of its children are, an empty placeholder node is used instead to preserve the octree's structure. 


\subsubsection{Map}

Usually, the \verb|map| function is used for generic data structures to create a projection for each element and returns a data structure of the same type with the projected elements. Since the octree is not generic, the \verb|map| function returns the projection for each node of the octree as an array of \verb|'T| where \verb|'T| is the type of the projection. 
The functions's definition looks as follows: 

\begin{lstlisting}[language = FSharp]
let map (projection : OctreeNode -> GridCell -> int[] -> 'T) (tree: Octree) : ('T[])= ...
\end{lstlisting}

Contrary to the filter function, the \verb|map| function does not return a new octree since the type of the projection must not necessarily be hierarchical. Therefore an array of \verb|'T| is returned instead. 


\subsubsection{Choose}

A similar task to filtering is the choose function. However, instead of returning the filtered values directly, the decision function also maps the values to the desired type \verb|'T|. The decision function uses the same input as the decision function for filtering. Its return type is a tuple that consists of a \verb|Option<'T>| and a \verb|bool|. The option type is used to introduce null-mechanics into the functional context. An option can either be \verb|Some value| or \verb|None|. If the \verb|choose| function's returned type is \verb|Some value|, its value is filtered, the \verb|bool| value decides whether or not to traverse the children. 
The function's definition looks as follows: 

\begin{lstlisting}[language = FSharp]
let choose (decisionFun : OctreeNode -> GridCell -> int[] -> Option<'T>*bool) (tree: Octree) :('T[]) = ...
\end{lstlisting}


\begin{figure}[h]
    \centering
    \subcaptionbox{ \label{fig:octreeFilter}}{%
        \includegraphics[width=0.55\textwidth]{Implementation/octreeFilter.png}%7
        }\par\medskip
    \subcaptionbox{ \label{fig:octreeMap}}{%
        \includegraphics[width=0.55\textwidth]{Implementation/octreeMap.png}%
        }\par\medskip        
    \subcaptionbox{ \label{fig:octreeChoose}}{%
        \includegraphics[width=0.55\textwidth]{Implementation/octreeChoose.png}%
        }
    \caption[Filter, map, and choose applied to an exemplary octree]
		{This figure shows an exemplary octree, on which the filter, map and choose function is applied. In (a) the filter function creates a new octree that only contains the nodes that are filtered. (b) shows the result of the map function. Each node in the octree is projected to a new type $t_{i}$ and the result is returned as an array. (c) shows the choose function. Only nodes are returned that fulfill the decision function and are projected to a new type $t_{i}$. }
    \label{fig:octreeFuns}
\end{figure}

Figure \ref{fig:octreeFuns} shows the described functions and its' effects on an exemplary octree. The \verb|filter| function returns a new octree, the\verb|map|  function returns an array of projected types, the \verb|choose| function is a combination of \verb|filter| and \verb|map| such that only those projections are returned that are of interest. 


\subsubsection{Replacing nodes}

As mutations introduce side effects that can lead to bugs, instead of changing information within an octree node, a new node is created containing the new data. The old node might still be in use in a different thread. Thus, race conditions are possible when mutating values. When information changes in an octree node, a new \verb|thunk| is created that stores the new content onto the disc, the old \verb|thunk| is discarded when it is not used anymore. The newly created node has a distinct position that is determined by the path in the octree. The octree is traversed recursively to find the location of the node. 
\\
When resolving the recursion, the content of all ancestor nodes changes as well, since one of the children is a new node. Therefore, for each ancestor, a new node is created too, containing the new content. Since the root changes as well, as the replaced node is a successor of it, a new octree is constructed each time a node is replaced. This octree, however, contains mostly the data of the old octree, except the replaced node. 

\begin{figure}[h]
    \centering
    \includegraphics[width=0.8\textwidth]{Implementation/octreeReplace.png}
    \caption[Example on replacing an octree node]
		{Replacing an octree node subsequently changes the node's ancestors as well. The left tree shows the original octree, the right tree is the new octree after the node [1,0] was replaced. The replaced node is highlighted with a red border, all nodes that changed due to the replacement are colored in orange. }
    \label{fig:octreeReplace}
\end{figure}

Figure \ref{fig:octreeReplace} shows an example on the replacement of a node. The nodes are labeled with their paths in the octree to identify them uniquely. The node with label \verb|[1,2]| is replaced. Thus, all ancestors in the octree are changed as well since the node got a new child. Nodes that changed are colored in orange. 


\subsection{Raycast}

An octree is an ideal data structure to accelerate ray casting. The hierarchical structure ensures that only a minimum set of nodes is tested for intersection. If the ray does not intersect the parent's bounding box, it does not intersect the children's bounding boxes as well. The raycast on the octree can be performed with logarithmic cost. The raycast is implemented using the octree's \verb|choose| function with a decision function that performs the intersection with the bounding box and returns a \verb|RaycastHit| structure containing all necessary information on the raycast hit. 


\subsection{Culling}

Culling utilizes the octree's \verb|filter| function to create a new octree that only contains nodes that are currently rendered. The culling function performs view-frustum culling as well as a \verb|level-of-detail| culling heuristic based on the node's size and distance to the near plane. Figure \ref{fig:octreeCulling} shows an exemplary culling performed on a tree. The \textit{level-of-detail} decreases for nodes that are further away from the camera and only those nodes are used that intersect the view frustum.

\begin{figure}
    \centering
    \includegraphics[width=0.8\textwidth]{Implementation/octreeCulling.png}
    \caption[Exemplary octree culling]
		{An exemplary culling is performed on a tree. Octree nodes that are darker with a thicker edge have a higher \textit{leve-of-detail}. Only close to the camera (red), the nodes with the highest \textit{level-of-detail} are rendered. Nodes that are outside the view frustum(orange) are discarded. }
    \label{fig:octreeCulling}
\end{figure}


Let $V$ be the volume of the node's bounding box, $win_x, win_y$ the dimensions of the window and $d_min$ be the smallest distance from the bounding box to the near plane in world space, clamped by the near and far plane. Furthermore, let $t$ be a user-controlled distance threshold. For this application, $t = 2$ is used. 

\begin{center}
$t_{win} = \frac{t}{max (win_x, win_y)}$ \\
$g = \frac{\sqrt[3]{V}}{d_{min}}$ \\
\end{center}
The decision whether or not to render the node is determined by:
\begin{center}
$g > t_{win}$
\end{center}

The user-controlled threshold $t$ is divided by the largest window dimension to allows the \textit{level-of-detail} the adapt to the window size. $d_min$ is calculated by projecting all corners of the node's bounding box to view space and negating the $z$ value. The smallest value is then clamped to the view frustum, so the computation stays in visible view space. A smaller $t$ win allows for higher \textit{level-of-detail} to be displayed. The granularity $g$ of an octree node is controlled by the minimal distance and the size of the node. Nodes close to the camera have higher the granularity. Smaller nodes have smaller granularity. Figure \ref{fig:targetpointDistance} 

\begin{figure}
    \centering
    \subcaptionbox{ \label{fig:targetpointDistance10}}{%
        \includegraphics[width=0.6\textwidth]{Implementation/targetpointDistance10.png}%7
        }\par\medskip
    \subcaptionbox{ \label{fig:targetpointDistance25}}{%
        \includegraphics[width=0.6\textwidth]{Implementation/targetpointDistance25.png}%
        }\par\medskip        
    \subcaptionbox{ \label{fig:targetpointDistance50}}{%
        \includegraphics[width=0.6\textwidth]{Implementation/targetpointDistance50.png}%
        }
    \caption[Different $t$ values for Technologiezentrum]
		{This figure shows the Technologiezentrum rendered with a $t$ value of 2, 5, and 10 in (a) - (c). A smaller $t$ results in a denser visualization of the point cloud for the current view. }
    \label{fig:targetpointDistance}
\end{figure}


\subsection{Diff}

As discussed in Section \ref{sec:renderHorizon}, the render horizon is the set of nodes that are still rendered, and at least one of the children is not. The \textit{Shape-Assisted Local Level-of-Detail Increment} interaction, described in Section \ref{sec:lod_increment}, uses the render horizon to collect a set of nodes that are beyond the horizon to draw additional points onto the screen. This set of nodes is obtained by the octree's \verb|diff| function. 
\\
Let $t1, t2$ be two octrees where $t2$ is a sub tree of $t1$. The function \verb|diff t1 t2| returns all nodes from t1, whose parents exist in t1 and t2, but the node itself only exists in t1.
\\
The \verb|diff| function takes nodes that are in the first octree and not in the second octree. In the case of the \textit{Shape-Assisted Local Level-of-Detail Increment}, interaction $t1$ is the complete octree, $t2$ is the culled octree that only contains nodes that are rendered. 
\\
Additionally, not only the children are collected, but depending on a depth value, the children's successors are collected as well. 


\section{Sequential Computation Applicator}

The octree receives changes on a regular basis from multiple sources in this application. The shape detection coroutine continuously inserts detected primitive shapes into the octree, resulting in a new octree every time. User interactions change the point set by selecting regions of interest. To synchronize the octree access, locking mechanics can be used. However, such mechanics can get confusing easily, especially when the application grows. 
\\
All changes to the octree are dependent on the state of the octree that, in the meantime, may be altered by an operation from a different thread. Furthermore, since a new octree is created every time, the result of an operation may be overwritten by a second operation that starts while the first operation has not yet finished. All subsequent operations depend on the result of the previous operation. Thus a structure is needed that handles such dependent operations. 
\\
The \verb|Sequential Computation Applicator| is a structure that provides a synchronized way of processing sequential operations. The applicator's functionality is synchronized such that multiple threads can dispatch operations on the octree. The \verb|Sequential Computation Applicator| is a wrapper around a \verb|ModRef<'T>| that invokes changes on the value regularly. 
In this case, the type of the applicator wraps around a \verb|ModRef<Octree>|. 

All dispatched operations are processed sequentially, and the \verb|ModRef|'s value is changed after an operation is completed, thus notifying the Aardvark engine. Since Aardvark evaluates adaptives lazily, operation simply get aggregated as long as no results are requested (i.e. rendering is never stalled). 
The applicator provides the interfaces to dispatch an operation, as well as to dispatch an operation with high priority to enqueue an operation on the front. 

\begin{lstlisting}[language = FSharp]

    member public this.DispatchPrioritized(computation : 'T -> 'T) (timeout : int) (timeoutCallback  : unit -> unit) : unit = ...

    member public this.Dispatch(computation : 'T -> 'T) (timeout : int) (timeoutCallback  : unit -> unit) : unit = ...
   
\end{lstlisting}

The type \verb|'T| is the generic type of the \verb|Sequential Computation Applicator|, in this case, it is \verb|Octree|. A computation is dispatched that takes a \verb|'T|, and projects it to a new \verb|'T| (i.e. all of the previously mentioned computations). The input parameter on execution is the current value of the octree. 
Additionally, an operation can be shut down after a defined timeout in milliseconds. If this is the case the \verb|timeoutCallback| is invoked to allow clean up and error handling. 


\section{Multi-Threaded Environment}
\label {sec:multithreading}

Multithreading can be achieved on multiple scales, depending on the tasks need. The application uses three primary threads that run in parallel. 
\\
The Aardvark rendering engine, combined with the \verb|IMod| evaluation system and user interactions build the main thread of the application. One pitfall of the \verb|IMod| system is that it only reacts to changes in the system, however, to invoke procedures after a particular time without changes is not possible directly (since that would by definition involve mutation). 
\\


The \textit{user-guided shape detection} in Section \ref{sec:user_guided_sd} is triggered when the camera has not changed for a particular amount of time. As the Aardvark \verb|IMod| system only reacts to changes, such no-changes must be invoked by a separate thread that contonously checks for changes. If a shape detection should be performed, this thread dispatches the computation to the applicator thread.
\\
The \textit{sequential computation applicator} performs changes on the octree. All tasks that are dispatched from multiple threads are executed in a sequential order in the background. The thread transacts the changes to the main thread once a computation has finished. 

\begin{figure}[b]
    \centering
    \includegraphics[width=0.6\textwidth]{Implementation/multiThreading.png}
    \caption[Overview on the multi-threaded environment of the application]
		{The main thread controls the rendering, mod evaluation and interactions. Point selections from interactions are handed to the applicator thread. The shape detection is invoked by a separate thread and dispatched on the applicator thread as well.  }
    \label{fig:multiThreading}
\end{figure}

As Figure \ref{fig:multiThreading} shows, multiple threads are dispatching computations to the applicator thread. However, only this thread transacts changes to the main thread. The shape detection invoker thread works independently of the main thread. 
\\

The second parallelization technique used in this thesis is task-based multithreading. The goal is to identify tasks that are executable in parallel for multiple instances of the same type, such as per-point or per-node computations. As long as shared resources are accessed as read-only, tasks can be executed in parallel without race conditions and the need for locking. The input is an array of elements and a mapping function. The mapping function is executed for each element of the input array using the \verb|System.Threading.Tasks.Parallel.For| statement (which is part of .NET). The return type is a new array containing the results of the computation for each element at the original position. The function's signature looks as follows: 
\\
\begin{lstlisting}[language = FSharp]
module Parallel = 
  let map (computation : 'T1->'T2)(array : 'T1[]) : 'T2[]= ...
\end{lstlisting}
The technique is realized as an parallel implementation of F\#s  \verb|map| function for arrays. 

\verb|Parallel.map| is used when point or node conversions are needed, such as projecting points to screen space or calculating the distances of points to a shape. 


\section{Point-Cloud Rendering}
\label{sec:rendering}

Modern point clouds often contain several billion points. Current GPUs are not able to handle such amounts of points. The culling heuristic in Section \ref{sec:renderHorizon} already reduces the octree to set of nodes that can be processed by the GPU and can be drawn in real time. However, the point data may still be located on the hard drive and must be loaded into memory. 
\\
The Aardvark platform already contains a \textit{level-of-detail} point cloud rendering system that can handle out-of-core datasets. The rendering system uses a cache to store nodes that were rendered previously. Once a frame is redrawn, all rendered nodes are collected, and for new nodes, the data is being loaded into the memory. New nodes are added to the cache, nodes that are not used for this frame, are kept in the cache until the memory consumption requires removal of unused data. For each node new to the cache, a queue entry is added for a worker thread to load the point data into memory. The frame is drawn even though not all nodes are loaded. The frame is redrawn once the outstanding data is loaded onto the GPU, thus popping effects occur when the view changes drastically. 
\\
\\
Mathematically, a point has no extent. Therefore it cannot be depicted directly. A common way to draw points is to interpret them as splats of certain size and shape. In this thesis, the points are represented by a sphere imposter. For each point, a camera-aligned quad is created in a geometry shader, whose size is the diameter of the sphere, pixels that are outside the radius of the sphere are discarded. 
\\
Popping effects occur when the view changes and overlapping imposters change their order. Giving the splat a depth displacement that follows a sphere, counters such popping effects. The splat is extruded in world space, such that the depth value follows the curvature of the sphere, causing the imposters to intersect correctly. Figure \ref{fig:point_sprites} shows a direct comparison of using spheres and circular splats. 


\begin{figure}
\centering
\subcaptionbox{ \label{fig:point_circles}}{%
  \includegraphics[width=0.48\textwidth]{Implementation/pointCircles.png}%7
  }
\subcaptionbox{ \label{fig:point_spheres}}{%
  \includegraphics[width=0.48\textwidth]{Implementation/pointSpheres.png}%
  }
  
\caption[Comparison of (a) circular splats and (b) sphere imposters]
{This figure shows a direct comparison of using (a) circular splats and (b) sphere imposters with depth displacement. In (b) the intersections between points are visible due to the spherical shape, whereas points that are little below other points are occluded almost entirely.}
\label{fig:point_sprites}
\end{figure}
\chapter{Results}
\label{chap:results}

-- die descriptions von den figures erweitern um einen konklusiven satz "`we can see the behavior x or improvement y"'

All results are obtained on a PC with an AMD FX-9590, 16 GB RAM and an AMD Radeon HD 7970. The datasets Technologiezentrum and JB\_Haus are a courtesy of rmData\cite{rmdata}. Section \ref{sec:shape_detection_results} shows benchmarks for the shape detection and verifies the capability of the user-guided shape detection to be performed in interactive time. Section \ref{sec:interaction_results} provides a set of screenshots that show the workflow of the user-guided interactions on a different data set. 


\section{Shape Detection Results}
\label{sec:shape_detection_results}

The interactions proposed in this thesis heavily rely on the ability to detect meaningful shapes for regions of interest within interactive time. Therefore it must be assured that the shape detection provides results within interactive time such that the detected shapes can be used immediately for interactions. Section \ref{sec:shape_detection_performance} discusses the performance obtained in the test computer. 


\subsection{Performance}

\label{sec:shape_detection_performance}

The goal of the \textit{user-guided shape detection} is to provide meaningful results within interactive time, such that detected shapes can be used to support interactions immediately. The capabilities of the implementation by Schnabel et al. \cite{schnabel-2007-software} are benchmarked on three different datasets: 

\begin{center}
\begin{tabular}{ l | r | r | r }
																	& \textbf{\#Points}	& \textbf{\#Nodes}	& \textbf{max. LoD} \\
    \hline
  JB\_Haus.pts                    & 620.722           & 440             	& 15 \\
  Technologiezentrum\_Teil1.pts   & 11.762.924        & 8863             	& 15 \\
  Synthetic\_Primitives.pts     	& 472.000           & 315               & 5 \\
    
\end{tabular}
\end{center}

The tests are performed without user interaction. Instead, all octree nodes are collected and fed into the shape detection routine sequentially to retrieve results for each node of the point cloud. Therefore, the shape detection is performed for the complete point cloud for each \textit{level-of-detail}. Each octree node contains at most 5000 points. The results are averaged over five runs. Table \ref{table:schnabel_benchmarks} shows the results averaged over all nodes. It can be seen that detecting planes only is significantly faster than detecting all types of primitive shapes. However, detecting all types of primitive shapes still produces results within interactive time. The synthetic point cloud is constructed from a set of primitives that are discretized to point sets, such that all types of primitives exist. 

\begin{table}
    \centering
    \begin{tabular}{ l || r | r | r || r | r | r}
            &\multicolumn{3}{c||}{\textbf{\#Shapes}} & \multicolumn{3}{c}{\textbf{Time (s)}}\\
            &\textbf{min} & \textbf{max} & \textbf{avg}  & \textbf{min} & \textbf{max} & \textbf{avg}  \\
            \hline
            JB\_Haus*           & 0 & 6  & 1.31 & 0.000026 & 0.578664 & 0.022396 \\
            JB\_Haus            & 0 & 6  & 1.40 & 0.000024 & 0.741384 & 0.093264 \\
            Technologiezentrum*	& 0 & 10 & 1.18 & 0.000023 & 0.440614 & 0.018781 \\
            Technologiezentrum  & 0 & 10 & 1.19 & 0.000024 & 0.991554 & 0.083180 \\
            Synthetic*          & 0 & 7  & 1.89 & 0.000026 & 0.440064 & 0.029779 \\
            Synthetic           & 0 & 7  & 1.24 & 0.000024 & 0.753345 & 0.136474 \\
        \end{tabular}
    \caption[Shape-detection performance measure for different datasets]
		{Performance measures for the different datasets averaged over all nodes. The number of shapes and duration are listed as minimum, maximum and average. Each dataset is benchmarked using plane detection only (items with *), as well as detecting all types of primitive shapes. }
    \label{table:schnabel_benchmarks}
\end{table}


\begin{figure}[h]
    \centering
    \includegraphics[width=1\textwidth]{Results/shapes_total_vs_lod.png}
    \caption[Plot of the total number of shapes vs. to the level-of-detail of the node]
		{Plot of the total number of shapes vs. the \textit{level-of-detail} of the node. All values are averaged over all nodes that share the same \textit{level-of-detail}.}
    \label{fig:shapes_total_vs_lod}
\end{figure}

\begin{figure}[h]
    \centering
    \includegraphics[width=1\textwidth]{Results/shapes_averaged_vs_lod.png}
    \caption[Plot of the average number of shapes per node vs. the \textit{level-of-detail} of the node]
		{Plot of the average number of shapes per node vs. the \textit{level-of-detail} of the node. All values are averaged over all nodes that share the same \textit{level-of-detail}.}
    \label{fig:shapes_averaged_vs_lod}
\end{figure}

\begin{figure}[h]
    \centering
    \includegraphics[width=1\textwidth]{Results/time_vs_lod.png}
    \caption[Plot of the number of shapes vs. the computation time]
		{Plot of the number of shapes vs. the computation time. All values are averaged over all nodes that share the same \textit{level-of-detail}.}
    \label{fig:time_vs_lod}
\end{figure}


Figure \ref{fig:shapes_total_vs_lod} shows the total number of shapes per \textit{level-of-detail}. All three datasets share an increase in the number of shapes in the third quarter before a rapid decline for the highest \textit{level-of-detail}. Figure \ref{fig:shapes_averaged_vs_lod} shows the number of shapes divided by the number of nodes per \textit{level-of-detail}. Both figures emphasize that the majority of shapes is not found in the highest \textit{level-of-detail}. Reasons for this is that many nodes only contain a handful of points or only contain a single structure. 
\\

Figure \ref{fig:time_vs_lod} shows the calculation time compared to the \textit{level-of-detail}. Shape detection for smaller, more dense nodes take less time than for larger nodes with lower \textit{level-of-detail} as in larger nodes, usually more shapes are found. 


\subsection{Problems and undesired Behavior}

A problem with the shape detection implementation by Schnabel et al. \cite{schnabel-2007-software} are reoccurring non-terminations for some octree nodes, causing the shape detection coroutine to stall. To circumvent this problem, all shape detection tasks that are dispatched to the \verb|sequential computation applicator| are assigned a timeout value of one second, after which the computation is interrupted. 
\\
Another reoccurring problem with the shape detection is the plausibility of detected shapes. The RANSAC options guarantee that shapes are found that fit the local geometry within a certain margin, $\alpha$ for the normal's angle and $\epsilon$ for the distance the shape. So within theses two parameters, the shape is considered to be valid. However, certain constellations of points allow the shape detection to produce non-plausible results that are accurate regarding the RANSAC parameters but are not plausible to the eye. A prominent example is a torus that is fitted onto a section that describes a cylinder. The major radius of the torus is of such dimensions that the local cylinder fits within the curvature of the torus. Thus, a torus is detected, rather than the simpler cylinder. Figure \ref{fig:missfittedTorus} shows this behavior within an example scene that consists multiple primitives. Figure \ref{fig:missfittedTorus2} shows the size of the detected torus. 
\\
Such non-plausible shapes can exist due to the density-controlled $\epsilon$ parameter that weakens the margin for octree nodes of larger volume. Even within a node, the density can strongly vary for different regions. 

\begin{figure}[h]
\centering
\subcaptionbox{ \label{fig:missfittedTorus1}}{%
  \includegraphics[width=0.49\textwidth]{Results/missfittedTorus1.png}%7
  }%\par\medskip
\subcaptionbox{ \label{fig:missfittedTorus2}}{%
  \includegraphics[width=0.49\textwidth]{Results/missfittedTorus2.png}%
  }      
\caption[Implausible torus is detected instead of a more plausible cylinder. ]
{This figure shows a cylinder from the synthetic point cloud whose points are classified as a torus instead of a cylinder. Even tough the points fit the torus, determined by the RANSAC options, the result is not plausible, since the user would expect a cylinder for this constellation of points. }
\label{fig:missfittedTorus}
\end{figure}


\subsection{Results}

This section presents results for the RANSAC shape detection performed on three data sets. The results in Figure \ref{fig:synthetic_point_cloud_results}, \ref{fig:JB_haus_results}, and \ref{fig:technologiezentrum_results} are obtained by segmenting the point clouds as a whole. Results for the interactive shape detection can be seen in Figure \ref{fig:technologiezentrum_interactive_shape_detection}. 

The synthetic test scene consists of two planes, three cylinders, three tori, and three cones.  Figure \ref{fig:synthetic_point_cloud_results} shows the synthetic point cloud, as well as the shapes that are detected within the point cloud. As the RANSAC approach randomly selects a subset of points, the results can be different for different runs. 

\begin{figure}[h]
\centering
\subcaptionbox{ \label{fig:synthetic_point_cloud}}{%
  \includegraphics[width=0.7\textwidth]{Results/synthetic_point_cloud.png}%7
  }
\subcaptionbox{ \label{fig:synthetic_point_cloud_shapes}}{%
  \includegraphics[width=0.7\textwidth]{Results/synthetic_point_cloud_shapes.png}%
  }
\caption[Rendering of the synthetic point cloud in (a), rendering of the detected shapes in (b)]
{(a) shows the synthetic point cloud, consisting of two planes, three cylinders, three tori, and three cones. (b) shows the detected shapes rendered as triangle meshes. For each shape in the point cloud, the RANSAC shape detection has found a suitable primitive shape. }
\label{fig:synthetic_point_cloud_results}
\end{figure}

\begin{figure}
\centering
\subcaptionbox{ \label{fig:jb_haus}}{%
  \includegraphics[width=\textwidth]{Results/jb_haus.png}%7
  }
\subcaptionbox{ \label{fig:jb_haus_shapes}}{%
  \includegraphics[width=\textwidth]{Results/jb_haus_shapes.png}%
  }
\caption[Rendering of JB\_haus in (a), rendering of the detected shapes in (b)]
{JB\_haus rendered as point cloud in (a), (b) shows all shapes detected in the point cloud. }
\label{fig:JB_haus_results}
\end{figure}

\begin{figure}
\centering
\subcaptionbox{ \label{fig:technologiezentrum}}{%
  \includegraphics[width=\textwidth]{Results/technologiezentrum.png}%7
  }
\subcaptionbox{ \label{fig:technologiezentrum_shapes}}{%
  \includegraphics[width=\textwidth]{Results/technologiezentrum_shapes.png}%
  }
\caption[[Rendering of the Technologiezentrum in (a), rendering of the detected shapes in (b)]]
{Technologiezentrum rendered as point cloud in (a), (b) shows all shapes detected in the point cloud. }
\label{fig:technologiezentrum_results}
\end{figure}

\begin{figure}
\centering
\subcaptionbox{ \label{fig:technologiezentrum_interactive_shape_detection1}}{%
  \includegraphics[width=\textwidth]{Results/technologiezentrum_interactive_shape_detection1.png}%7
  }
\subcaptionbox{ \label{fig:technologiezentrum_interactive_shape_detection2}}{%
  \includegraphics[width=\textwidth]{Results/technologiezentrum_interactive_shape_detection2.png}%
  }
\caption[Two examples of user-guided shape detection]
{Based on the cursor's position a different shapes are selected. (a) and (b) show different shapes for different parts of the point cloud. Both shapes are a part of a wall. }
\label{fig:technologiezentrum_interactive_shape_detection}
\end{figure}


\section{Interaction Results}
\label{sec:interaction_results}

This section presents a set of figures that showcase the different interactions from Section \ref{sec:interactions}. The interactions are performed on the Technologiezentrum dataset since the JB\_Haus point cloud is used as an example throughout this thesis already. 

\begin{figure}[h]
\centering
\subcaptionbox{ \label{fig:technologiezentrum_lasso1}}{%
  \includegraphics[width=0.9\textwidth]{Results/technologiezentrum_lasso1.png}%7
  }
\subcaptionbox{ \label{fig:technologiezentrum_lasso2}}{%
  \includegraphics[width=0.9\textwidth]{Results/technologiezentrum_lasso2.png}%
  }
\caption[Example of an improved lasso selection]
{A lasso selection is performed on the selected support shape in (a). Only points are selected that lie on the support shape as shown in (b). Point in front and back of the support shape are not selected. }
\label{fig:technologiezentrum_lasso}
\end{figure}


\begin{figure}
\centering
\subcaptionbox{ \label{fig:technologiezentrum_brush1}}{%
  \includegraphics[width=\textwidth]{Results/technologiezentrum_brush1.png}%7
  }
\subcaptionbox{ \label{fig:technologiezentrum_brush2}}{%
  \includegraphics[width=\textwidth]{Results/technologiezentrum_brush2.png}%
  }
\caption[Example of an improved volumetric brush selection]
{A volumetric brush selection is performed on the selected support shape in (a). Point are only selected if they belong to the support shape and intersect the brush. The result of the selection can be seen in (b).}
\label{fig:technologiezentrum_brush}
\end{figure}


\begin{figure}
\centering
\subcaptionbox{ \label{fig:technologiezentrum_lod_increment1}}{%
  \includegraphics[width=\textwidth]{Results/technologiezentrum_lod_increment1.png}%7
  }
\subcaptionbox{ \label{fig:technologiezentrum_lod_increment2}}{%
  \includegraphics[width=\textwidth]{Results/technologiezentrum_lod_increment2.png}%
  }
\caption[Example of the local increment of level-of-detail]
{The \textit{level-of-detail} is incremented along the support shape. (a) shows the original rendering model of the point cloud, (b) shows the point cloud with additional points. }
\label{fig:technologiezentrum_lod_increment}
\end{figure}





\chapter{Conclusion and Future Work}
\label{chap:conclusion}


This chapter completes this thesis by presenting a conclusion on the work and proposing future work.

\subsection{Out-of-core Representation of Point Clouds}

Modern point clouds often consist of several billion points and consume several gigabytes of memory. They are simply too large to fit into system memory as a whole. This thesis describes a functional octree structure that can handle the out-of-core functionality of the point cloud. Data is stored in chunks in a file and is loaded into system memory when needed. Additionally, the point cloud's octree nodes contain a subset of points from its children, thus creating a multiscale representation of the point cloud that is used for rendering and interactions. 


\subsection{User-guided Shape Detection}

This thesis shows an alternative use of the shape detection algorithm by Schnabel et al.\cite{schnabel-2007-efficient} which lets the user control the regions that are segmented. By pointing to a region with the mouse, the system selects the most suitable octree node to be segmented. The number of points in this node is chosen in such a way that the shape detection delivers results in interactive time. This immediate feedback allows the user to use the detected shapes as support shapes for improved interactions. 


\subsection{Improvement of Interaction Techniques}

Several improvements to well-known two-dimensional interaction techniques are discussed in this thesis. We propose a way of pre-filtering point for interactions, such that only points are considered that follow the curvature of a primitive shape that is picked by the user. Classic Point Picking is improved such that only points are picked that belong to the support shape. Only using points that belong to a shape is especially useful when trying to pick points that are otherwise occluded or lie on the edge of a structure. Multiple ways exist to select regions in a point cloud. We utilize a support shape to improve the Lasso Selection. Classic Lasso Selection selects all points, whose projection lies inside the lasso on the near plane. The user selects 'through' the point cloud, whereas when using a support shape, the user only selects points that lie on this support shape. The volumetric brush allows the user to select points that are in the foreground. Instead of consulting the depth buffer to retrieve the brush's position the position of the cursor on the support shape is used. Thus the brush does not follow the structures that are in the foreground but the curvature of the support shape. Therefore points can be selected that are occluded otherwise. 


\subsection{Novel Interaction to magnify Local Details}

The last task of this thesis was to design a novel interaction technique called \textit{Shape-Assisted Local Level-of-Detail Increment}. The \textit{level-of-detail} rendering of the point cloud only displays a set of points such that the graphics card can render the points in real time. Thus, details may get culled and are not rendered. The interaction technique collects points from nodes that are not rendered due to their \textit{level-of-detail}. These additional points follow the curvature of the support shape. Along the support shape, extra points are rendered that would otherwise be culled, therefore, allowing the user to get a more detailed look at structures. 


\section{Future Work}

As the focus of this thesis was the design of the user-guided shape detection and the assisted interactions, future work will focus on performance and robustness improvements on various fronts. The usage of a selection data structure, such as a Selection Octree\cite{scheiblauer2011out}, can improve the performance when selecting or editing the point cloud. Also, different ways can be explored to improve computation speed by taking advantage of the parallel architecture of the graphics card. 
\\

As the selection processing is performed asynchronously, feedback is not displayed immediately, but with a delay. To overcome this gap between interaction and result, a visual selection can be performed on the GPU as presented in \cite{rainer2016visual}. This technique utilizes volumetric shadows to create a visual selection using the GPU's stencil buffer. 
\\

Shape detection is performed using an external library by Schnabel et al.\cite{schnabel-2007-software}. Multiple problems occur, such as non-termination or non-plausible shape matching. More work must be carried out to limit or eliminate these problems. The robustness of the shape detection, especially when detecting non-planar primitive shapes, suffers due to weak constraints. Alternatively, the shape detection can be implemented in such a way, that particular types of shapes are prioritized to reduce the amount of non-plausible shapes. 
\\



\backmatter

% Use an optional list of figures.
\listoffigures % Starred version, i.e., \listoffigures*, removes the toc entry.

% Use an optional list of tables.
\cleardoublepage % Start list of tables on the next empty right hand page.
\listoftables % Starred version, i.e., \listoftables*, removes the toc entry.

% Use an optional list of alogrithms.
\listofalgorithms
\addcontentsline{toc}{chapter}{List of Algorithms}

% Add an index.
\printindex

% Add a glossary.
\printglossaries

% Add a bibliography.
\bibliographystyle{alpha}
\bibliography{intro}

\end{document}