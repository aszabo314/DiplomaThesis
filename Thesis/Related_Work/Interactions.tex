\section {Interactions}
% % Umschreiben!!!Data selection is a fundamental task in visualization because it serves as a pre-requisite to many follow-up interactions.


Demir et. al\cite{demir2015procedural} present a framework for semi-automatic point cloud segmentation and procedural modeling. Interactive editing tools to let the user create new point clouds using procedural copy and paste operations, as well as smart resizing. 
\\
\\
O-Snap by Arikan et al. \cite{arikan-2013-osn} utilizes Schnabel's algorithm to extract an initial model from a point cloud used in a reconstruction and modeling pipeline. Local adjacency relations of the extracted shapes are used in an interactive workflow to snap polygon elements together, while simultaneously fitting the input point cloud to ensure the planarity of the polygons. 
\\
\\
The lasso interaction is a common tool to select regions in 2d. Yu et. al\cite{yu2012efficient} presents two new methods of interaction using only two-dimensional input. The result are two methods that turn a two-dimensional lasso into a three-dimensional volume that is fitted to spatial structure of the point cloud. Similar to sketch-based modeling\cite{igarashi2007teddy}, TeddySelection inflates a user-drawn lasso using a heuristic that takes the local point density into account and fits it to the indented region. CloudLasso uses the Marching Cubes algorithm\cite{lorensen1987marching} to identify and select regions inside the lasso where the density is above a threshold. Both techniques only use two degrees of freedom, thus can be used in a traditional mouse-based interaction, as well as in direct-touch environments. 

\begin{figure}
    \centering
    \includegraphics[width=0.81\textwidth]{Related_Work/teddyCloudSelection.png}%7
    \caption{The region is grey describes the selection volume. Points that are selected are colored in pink. (a) shows a simple lasso selection, (b) TeddySelection and (c) shows the CloudLasso. Image by Yu et. al\cite{yu2012efficient}.}
    \label{fig:teddyCloudSelection}
\end{figure}

Figure \ref{fig:teddyCloudSelection} compares the results of a simple lasso selection, TeddySelection and the CloudLasso. 




An example for a three-dimensional interaction technique is the Volumetric Brush by Weyrich et. al\cite{weyrich2004post}. The bush follows the local curvature by retrieving the current depth value for the cursor's position from the zBuffer. The reconstructed world space position is then used to as center of a volume, usually a sphere, to select points. 
