\section{Motivation \& Problem Definition}

With the increasing availability of 3d sensors over the past years, segmentation of point-clouds has received increasing interest as well. While different methods of generating point-cloud data exist, such as laser scanners, Microsoft Kinect or photogrammetic reconstructions, the presented data however, commonly lacks structure and semantic information. The objective of segmentation is to find homogeneous regions in point-clouds with similar characteristics to help the user understanding local and global structures. Moreover, it can be used to introduce semantic information into a unstructured data, providing the user with more interaction possibilities. Current solutions, such as presented in \cite{schnabel-2007-efficient}, \cite{schnabel-2007-ransac} can already produce precise segmentations of point-clouds. However, the runtime of these algorithms increases with the size of the point-cloud, so producing results for billions of points is not feasible in real-time. However, when looking at raw numbers, the approach by Schnabel et al. \cite{schnabel-2007-ransac} delivers promising results for point-clouds of smaller size (\textless 12.000 points) in a fraction of a second. 
\\
Point-cloud datasets have grown in size at such a rapid pace that they are now simply too large to fit into system memory, yet alone graphics card memory. Therefore new solutions for out-of-core representations have emerged. The point-cloud is stored in a cached file on the hard drive and therefore cannot be accessed directly. Based on a culling heuristic, chunks of point-cloud data are loaded into memory when being processed or rendered. This continuous swapping of data yields the memory bandwidth as potential performance bottleneck when it comes to performance.
\\
This thesis presents an interactive approach to detect shapes on different scales using a \textbf{L}evel-\textbf{o}f-\textbf{D}etail (LoD) representation of a point-cloud. Instead of segmenting the whole point-cloud at once, the user's input is used to extract regions of interest that can be segmented within a fraction of a second, thus providing the user with immediate geometrical information on the currently focused region. Moreover, selecting points or regions in point-clouds can be tedious and cumbersome. This thesis presents several interactions that utilize the geometric information in order to streamline the user's workflow. 